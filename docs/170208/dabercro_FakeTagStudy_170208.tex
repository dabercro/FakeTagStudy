\documentclass{beamer}

\author[D. Abercrombie]{
  Daniel Abercrombie
}

\title{\bf \sffamily Fake Tagging Update (and Trends)}
\date{\today}

\usecolortheme{dove}

\usepackage[absolute,overlay]{textpos}
\usefonttheme{serif}
\usepackage{appendixnumberbeamer}
\usepackage{isotope}
\usepackage{hyperref}
\usepackage[english]{babel}
\usepackage{amsmath}
\setbeamerfont{frametitle}{size=\Large,series=\bf\sffamily}
\setbeamertemplate{frametitle}[default][center]
\usepackage{siunitx}
\usepackage{tabularx}
\usepackage{makecell}

\setbeamertemplate{navigation symbols}{}
\usepackage{graphicx}
\usepackage{color}
\setbeamertemplate{footline}[text line]{\parbox{1.083\linewidth}{\footnotesize \hfill \insertshortauthor \hfill \insertpagenumber /\inserttotalframenumber}}
\setbeamertemplate{headline}[text line]{\parbox{1.083\linewidth}{\footnotesize \hspace{-0.083\linewidth} \textcolor{blue}{\sffamily \insertsection \hfill \insertsubsection}}}

\IfFileExists{/Users/dabercro/GradSchool/Presentations/MIT-logo.pdf}
             {\logo{\includegraphics[height=0.5cm]{/Users/dabercro/GradSchool/Presentations/MIT-logo.pdf}}}
             {\logo{\includegraphics[height=0.5cm]{/home/dabercro/MIT-logo.pdf}}}

\usepackage{changepage}

\newcommand{\beginbackup}{
  \newcounter{framenumbervorappendix}
  \setcounter{framenumbervorappendix}{\value{framenumber}}
}
\newcommand{\backupend}{
  \addtocounter{framenumbervorappendix}{-\value{framenumber}}
  \addtocounter{framenumber}{\value{framenumbervorappendix}}
}

\graphicspath{{figs/}}

\begin{document}

\begin{frame}[nonumbering]
  \titlepage
\end{frame}

\begin{frame}
  \frametitle{Introduction}

  The most recent measurement scale factor of signal-like fat jets is presented
  \href{http://t3serv001.mit.edu/~dabercro/docs/WTagStudy/dabercro_WTagStudy_170127.pdf}
       {\textcolor{blue}{here}}.
  To compliment this, we present a scale factor of background-like fat jets.
  To get a background-enriched sample of fat jets, a $\gamma + $jets selectionis used.
  Drell Yan selections in both $Z\rightarrow\mu\mu$ and $Z\rightarrow ee$ channels are used
  as a systematic, but these channels include the irreducible, signal-like background
  that is the di-boson process.

  \vspace{12pt}

  Largely, this is a luminosity update of
  \href{http://t3serv001.mit.edu/~dabercro/docs/FakeTagStudy/dabercro_FakeTagStudy_161028.pdf}
       {\textcolor{blue}{a previous study}}
  measuring the fake-tag scale factor.
  In addition, recoil and pileup dependence are considered in order to keep this interesting.

\end{frame}

\begin{frame}
  \frametitle{Samples and Selection}

  \begin{itemize}
  \item Using mono-jet samples and selection for control regions
  \item Main fake tagging is done using $\gamma + $jets selection
  \item Both Di-lepton regions are separately used for systematic
  \item Do not have the triggers (updated Nero has different list) or trigger efficiencies yet
  \end{itemize}

\end{frame}

\begin{frame}
  \frametitle{Tagging Variable Distributions}

  Shape of Pruned mass and $\tau_2/\tau_1$ before V-tagging cuts are applied.

  \vspace{12pt}

  \begin{columns}
    \begin{column}{0.5\linewidth}
      \includegraphics[width=\linewidth]{170208_FakeTagStudy/nocut_gjets_fatjet1PrunedML2L3.pdf}
    \end{column}
    \begin{column}{0.5\linewidth}
      \includegraphics[width=\linewidth]{170208_FakeTagStudy/nocut_gjets_fatjet1tau21.pdf}
    \end{column}
  \end{columns}

\end{frame}

\begin{frame}
  \frametitle{Tagging Variable Distributions After Cut}

  Cutting on one variable shapes the other

  \vspace{12pt}

  \begin{columns}
    \begin{column}{0.5\linewidth}
      \includegraphics[width=\linewidth]{170208_FakeTagStudy/full_gjets_fatjet1PrunedML2L3.pdf}
    \end{column}
    \begin{column}{0.5\linewidth}
      \includegraphics[width=\linewidth]{170208_FakeTagStudy/full_gjets_fatjet1tau21.pdf}
    \end{column}
  \end{columns}

\end{frame}

\begin{frame}
  \frametitle{Scale Factor}

  The V-tagging cuts are: \\
  \textcolor{red}{$\SI{65}{GeV} < m_{SD} < \SI{105}{GeV}$ and $\tau_2/\tau_1 < 0.6$}

  \vspace{12pt}

  \begin{adjustwidth}{-1.5em}{-1.5em}
    {\tiny
      \centering

      \begin{tabular}{c|c|c|c|c}
        \hline
        & No Cut & $m_\text{pruned}$ & $\tau_2/\tau_1$ & V-tag cut \\
        \hline
        \makecell{Background \\ Subtracted \\ Data} & 248045.70 $\pm$ 498.38 & 30608.27 $\pm$ 175.45 & 56364.06 $\pm$ 237.95 & 21954.26 $\pm$ 148.74 \\
        \makecell{Signal-\\ matched MC} & 234409.46 $\pm$ 1347.84 & 31817.83 $\pm$ 489.35 & 56953.68 $\pm$ 632.26 & 22128.83 $\pm$ 417.37 \\
        \hline
        \makecell{Normalized \\ Ratio} & 1.00 $\pm$ 0.01 & 0.91 $\pm$ 0.01 & 0.94 $\pm$ 0.01 & \fcolorbox{red}{yellow}{0.94 $\pm$ 0.02} \\
        \hline
      \end{tabular}
    }
  \end{adjustwidth}

\end{frame}

\begin{frame}
  \frametitle{Systematics}

  $\gamma + $jets selection is almost entirely background-like, so there's no need for subtraction.
  Systematics are evaluated by the following:

  \begin{itemize}
  \item Jet scaling/smearing hybrid method
  \item Replacing Purity sample with QCD
  \item Using di-lepton regions with background subtraction
  \end{itemize}

\end{frame}

\begin{frame}
  \frametitle{Jet Smearing and QCD Sample}

  Jet Smearing (this is Up, all are 1\% different):

  \vspace{12pt}

  \begin{adjustwidth}{-1.5em}{-1.5em}
    {\tiny
      \centering

      \begin{tabular}{c|c|c|c|c}
        \hline
        & No Cut & $m_\text{pruned}$ & $\tau_2/\tau_1$ & V-tag cut \\
        \hline
        \makecell{Background \\ Subtracted \\ Data} & 355452.39 $\pm$ 596.58 & 30607.11 $\pm$ 175.47 & 56361.32 $\pm$ 237.98 & 21953.21 $\pm$ 148.77 \\
        \makecell{Signal-\\ matched MC} & 325388.45 $\pm$ 1721.69 & 32332.71 $\pm$ 495.06 & 57747.21 $\pm$ 640.03 & 21241.38 $\pm$ 407.79 \\
        \hline
        \makecell{Normalized \\ Ratio} & 1.00 $\pm$ 0.01 & 0.87 $\pm$ 0.01 & 0.89 $\pm$ 0.01 & \fcolorbox{red}{yellow}{0.95 $\pm$ 0.02} \\
        \hline        
      \end{tabular}
    }
  \end{adjustwidth}

  Using the QCD Sample:

  \vspace{12pt}

  \begin{adjustwidth}{-1.5em}{-1.5em}
    {\tiny
      \centering

      \begin{tabular}{c|c|c|c|c}
        \hline
        & No Cut & $m_\text{pruned}$ & $\tau_2/\tau_1$ & V-tag cut \\
        \hline
        \makecell{Background \\ Subtracted \\ Data} & 248046.17 $\pm$ 498.38 & 30608.50 $\pm$ 175.45 & 56364.40 $\pm$ 237.95 & 21954.48 $\pm$ 148.74 \\
        \makecell{Signal-\\ matched MC} & 235441.56 $\pm$ 1439.00 & 32523.77 $\pm$ 534.91 & 57710.89 $\pm$ 676.34 & 22521.59 $\pm$ 451.38 \\
        \hline
        \makecell{Normalized \\ Ratio} & 1.00 $\pm$ 0.01 & 0.89 $\pm$ 0.02 & 0.93 $\pm$ 0.01 & \fcolorbox{red}{yellow}{0.93 $\pm$ 0.02} \\
        \hline
      \end{tabular}
    }
  \end{adjustwidth}

\end{frame}

\begin{frame}
  \frametitle{Using Di-Lepton for Systematics}

  The Z + jets region has a different quark-gluon fraction than $\gamma + $jets.
  We use $\gamma + $jets for the primary measurement since there are no signal-like
  backgrounds contaminating that would need to be subtracted.

  \vspace{6pt}

  \begin{columns}
    \begin{column}{0.5\linewidth}
      $Z \rightarrow \mu\mu$ \\
      \includegraphics[width=\linewidth]{170208_FakeTagStudy/full_Zmm_fatjet1PrunedML2L3.pdf}
    \end{column}
    \begin{column}{0.5\linewidth}
      $Z \rightarrow ee$ \\
      \includegraphics[width=\linewidth]{170208_FakeTagStudy/full_Zee_fatjet1PrunedML2L3.pdf}
    \end{column}
  \end{columns}

\end{frame}

\begin{frame}
  \frametitle{Di-Leption Selections}

  $Z \rightarrow \mu\mu$

  \vspace{12pt}

  \begin{adjustwidth}{-1.5em}{-1.5em}
    {\tiny
      \centering

      \begin{tabular}{c|c|c|c|c}
        \hline
        & No Cut & $m_\text{pruned}$ & $\tau_2/\tau_1$ & V-tag cut \\
        \hline
        \makecell{Background \\ Subtracted \\ Data} & 16766.04 $\pm$ 133.49 & 1933.95 $\pm$ 47.79 & 3705.24 $\pm$ 65.01 & 1370.71 $\pm$ 40.86 \\
        \makecell{Signal-\\ matched MC} & 15474.23 $\pm$ 71.67 & 2091.49 $\pm$ 28.06 & 3736.36 $\pm$ 35.28 & 1451.61 $\pm$ 23.51 \\
        \hline
        \makecell{Normalized \\ Ratio} & 1.00 $\pm$ 0.01 & 0.85 $\pm$ 0.02 & 0.92 $\pm$ 0.02 & \fcolorbox{red}{yellow}{0.87 $\pm$ 0.03} \\
        \hline
      \end{tabular}
    }
  \end{adjustwidth}

  $Z \rightarrow ee$

  \vspace{12pt}

  \begin{adjustwidth}{-1.5em}{-1.5em}
    {\tiny
      \centering

      \begin{tabular}{c|c|c|c|c}
        \hline
        & No Cut & $m_\text{pruned}$ & $\tau_2/\tau_1$ & V-tag cut \\
        \hline
        \makecell{Background \\ Subtracted \\ Data} & 11169.32 $\pm$ 109.45 & 1307.34 $\pm$ 39.45 & 2556.89 $\pm$ 53.89 & 927.54 $\pm$ 34.06 \\
        \makecell{Signal-\\ matched MC} & 10938.41 $\pm$ 60.01 & 1500.08 $\pm$ 23.68 & 2648.47 $\pm$ 29.95 & 1009.61 $\pm$ 18.62 \\
        \hline
        \makecell{Normalized \\ Ratio} & 1.00 $\pm$ 0.01 & 0.85 $\pm$ 0.03 & 0.95 $\pm$ 0.02 & \fcolorbox{red}{yellow}{0.90 $\pm$ 0.04} \\
        \hline
      \end{tabular}
    }
  \end{adjustwidth}

\end{frame}

\begin{frame}
  \frametitle{Systematic Uncertainties}
  We assume systematic uncertainties are symmetric for now
  \begin{center}
  \begin{tabular}{l|c}
    Source & Uncertainty \\
    \hline
    Jet Smearing & 1\% \\
    QCD vs. Purity & 1\% \\
    Di-Lepton Regions & 7\% \\
  \end{tabular}
  \end{center}
  Adding in Quadrature: \boxed{$\pm 7.1\%$}
  \vspace{12pt}
\end{frame}

\begin{frame}
  \frametitle{NPV Dependence}

  \begin{center}

    NPV 0 to 10: 0.99 $\pm$ 0.06 \\
    NPV 10 to 20: 0.95 $\pm$ 0.02 \\
    NPV 20 to 30: 0.90 $\pm$ 0.03 \\ 
    NPV 30 to 40: 0.94 $\pm$ 0.16

  \end{center}

  Not obvious trend, like we had seen before
  \href{http://t3serv001.mit.edu/~dabercro/docs/FakeTagStudy/dabercro_MonoV_161109.pdf}
       {\textcolor{blue}{here (slide 8)}},
  but probably shouldn't be ruled out.

\end{frame}

\begin{frame}
  \frametitle{$p_T$ Dependence}

  \begin{center}

    $p_T$ 250 to \SI{350}{GeV}: 0.93 $\pm$ 0.02 \\
    $p_T$ 350 to \SI{500}{GeV}: 0.94 $\pm$ 0.03 \\
    $p_T$ 550 to \SI{700}{GeV}: 1.00 $\pm$ 0.06 \\
    $p_T$ 700 to \SI{1000}{GeV}: 1.17 $\pm$ 0.15 \\

  \end{center}

  Good chance of a trend here.

\end{frame}

\begin{frame}
  \frametitle{Conclusions}

  \begin{itemize}
  \item Scale Factor: \boxed{$0.94 \pm 0.02 \pm 0.07$}, consistent with 1
  \item Need trigger list and scale factors
  \item There may be a strong dependence on NPV and $p_T$
    \begin{itemize}
    \item For some reason last NPV bin has worse stats (reweighting?)
    \item Will produce histograms showing trends
    \end{itemize}
  \end{itemize}

\end{frame}

\beginbackup

\begin{frame}
  \frametitle{Backup Slides}
\end{frame}

\begin{frame}
   \frametitle{\small 161109\_1/tight\_Zee\_fatjet1tau21}
   \centering
   \includegraphics[width=0.7\linewidth]{161109_1/tight_Zee_fatjet1tau21.pdf}
\end{frame}

\begin{frame}
   \frametitle{\small 161109\_1/nocut\_Zee\_fatjet1tau21}
   \centering
   \includegraphics[width=0.7\linewidth]{161109_1/nocut_Zee_fatjet1tau21.pdf}
\end{frame}

\begin{frame}
   \frametitle{\small 161109\_1/tight\_signal\_fatjet1tau21}
   \centering
   \includegraphics[width=0.7\linewidth]{161109_1/tight_signal_fatjet1tau21.pdf}
\end{frame}

\begin{frame}
   \frametitle{\small 161026/photon\_full\_fatjet1tau21}
   \centering
   \includegraphics[width=0.7\linewidth]{161026/photon_full_fatjet1tau21.pdf}
\end{frame}

\begin{frame}
   \frametitle{\small 161028/photon\_full\_fatjet1tau21}
   \centering
   \includegraphics[width=0.7\linewidth]{161028/photon_full_fatjet1tau21.pdf}
\end{frame}

\begin{frame}
   \frametitle{\small 160802/dilep\_full\_fatjet1tau21}
   \centering
   \includegraphics[width=0.7\linewidth]{160802/dilep_full_fatjet1tau21.pdf}
\end{frame}

\begin{frame}
   \frametitle{\small 161109\_1/tight\_Zmm\_fatjet1tau21}
   \centering
   \includegraphics[width=0.7\linewidth]{161109_1/tight_Zmm_fatjet1tau21.pdf}
\end{frame}

\begin{frame}
   \frametitle{\small 161109\_1/nocut\_Zmm\_fatjet1tau21}
   \centering
   \includegraphics[width=0.7\linewidth]{161109_1/nocut_Zmm_fatjet1tau21.pdf}
\end{frame}

\begin{frame}
   \frametitle{\small 161109\_1/tight\_Wen\_fatjet1tau21}
   \centering
   \includegraphics[width=0.7\linewidth]{161109_1/tight_Wen_fatjet1tau21.pdf}
\end{frame}

\begin{frame}
   \frametitle{\small 161109\_1/nocut\_Wen\_fatjet1tau21}
   \centering
   \includegraphics[width=0.7\linewidth]{161109_1/nocut_Wen_fatjet1tau21.pdf}
\end{frame}

\begin{frame}
   \frametitle{\small 161109\_1/tight\_Wmn\_fatjet1tau21}
   \centering
   \includegraphics[width=0.7\linewidth]{161109_1/tight_Wmn_fatjet1tau21.pdf}
\end{frame}

\begin{frame}
   \frametitle{\small 161109\_1/nocut\_Wmn\_fatjet1tau21}
   \centering
   \includegraphics[width=0.7\linewidth]{161109_1/nocut_Wmn_fatjet1tau21.pdf}
\end{frame}

\begin{frame}
   \frametitle{\small 161109\_1/tight\_gjets\_fatjet1tau21}
   \centering
   \includegraphics[width=0.7\linewidth]{161109_1/tight_gjets_fatjet1tau21.pdf}
\end{frame}

\begin{frame}
   \frametitle{\small 161026/photon\_nocut\_fatjet1tau21}
   \centering
   \includegraphics[width=0.7\linewidth]{161026/photon_nocut_fatjet1tau21.pdf}
\end{frame}

\begin{frame}
   \frametitle{\small 161027/photon\_nocut\_fatjet1tau21}
   \centering
   \includegraphics[width=0.7\linewidth]{161027/photon_nocut_fatjet1tau21.pdf}
\end{frame}

\begin{frame}
   \frametitle{\small 161028/photon\_nocut\_fatjet1tau21}
   \centering
   \includegraphics[width=0.7\linewidth]{161028/photon_nocut_fatjet1tau21.pdf}
\end{frame}

\begin{frame}
   \frametitle{\small 160802/dilep\_nocut\_fatjet1tau21}
   \centering
   \includegraphics[width=0.7\linewidth]{160802/dilep_nocut_fatjet1tau21.pdf}
\end{frame}

\begin{frame}
   \frametitle{\small 160726/semilep\_full\_massp\_tau21\_fatjettau21}
   \centering
   \includegraphics[width=0.7\linewidth]{160726/semilep_full_massp_tau21_fatjettau21.pdf}
\end{frame}

\begin{frame}
   \frametitle{\small 160726\_background/semilep\_full\_massp\_tau21\_fatjettau21}
   \centering
   \includegraphics[width=0.7\linewidth]{160726_background/semilep_full_massp_tau21_fatjettau21.pdf}
\end{frame}

\begin{frame}
   \frametitle{\small 160726\_midbackground/semilep\_full\_massp\_tau21\_fatjettau21}
   \centering
   \includegraphics[width=0.7\linewidth]{160726_midbackground/semilep_full_massp_tau21_fatjettau21.pdf}
\end{frame}

\begin{frame}
   \frametitle{\small 160726\_morebackground/semilep\_full\_massp\_tau21\_fatjettau21}
   \centering
   \includegraphics[width=0.7\linewidth]{160726_morebackground/semilep_full_massp_tau21_fatjettau21.pdf}
\end{frame}

\begin{frame}
   \frametitle{\small 160726\_background/semilep\_full\_fatjettau21}
   \centering
   \includegraphics[width=0.7\linewidth]{160726_background/semilep_full_fatjettau21.pdf}
\end{frame}

\begin{frame}
   \frametitle{\small 160726\_midbackground/semilep\_full\_fatjettau21}
   \centering
   \includegraphics[width=0.7\linewidth]{160726_midbackground/semilep_full_fatjettau21.pdf}
\end{frame}

\begin{frame}
   \frametitle{\small 160726\_morebackground/semilep\_full\_fatjettau21}
   \centering
   \includegraphics[width=0.7\linewidth]{160726_morebackground/semilep_full_fatjettau21.pdf}
\end{frame}

\begin{frame}
   \frametitle{\small 160727\_down/semilep\_full\_fatjettau21}
   \centering
   \includegraphics[width=0.7\linewidth]{160727_down/semilep_full_fatjettau21.pdf}
\end{frame}

\begin{frame}
   \frametitle{\small 160726\_up/semilep\_full\_fatjettau21}
   \centering
   \includegraphics[width=0.7\linewidth]{160726_up/semilep_full_fatjettau21.pdf}
\end{frame}

\begin{frame}
   \frametitle{\small 160726/semilep\_full\_highpt\_fatjettau21}
   \centering
   \includegraphics[width=0.7\linewidth]{160726/semilep_full_highpt_fatjettau21.pdf}
\end{frame}

\begin{frame}
   \frametitle{\small 160726/semilep\_nocut\_nsmalljets\_fatjetDPhiLep1}
   \centering
   \includegraphics[width=0.7\linewidth]{160726/semilep_nocut_nsmalljets_fatjetDPhiLep1.pdf}
\end{frame}

\begin{frame}
   \frametitle{\small 161109\_1/full\_Zee\_fatjet1PrunedML2L3}
   \centering
   \includegraphics[width=0.7\linewidth]{161109_1/full_Zee_fatjet1PrunedML2L3.pdf}
\end{frame}

\begin{frame}
   \frametitle{\small 161109\_1/tight\_Zee\_fatjet1PrunedML2L3}
   \centering
   \includegraphics[width=0.7\linewidth]{161109_1/tight_Zee_fatjet1PrunedML2L3.pdf}
\end{frame}

\begin{frame}
   \frametitle{\small 161109\_1/nocut\_Zee\_fatjet1PrunedML2L3}
   \centering
   \includegraphics[width=0.7\linewidth]{161109_1/nocut_Zee_fatjet1PrunedML2L3.pdf}
\end{frame}

\begin{frame}
   \frametitle{\small 161026/photon\_full\_fatjet1PrunedML2L3}
   \centering
   \includegraphics[width=0.7\linewidth]{161026/photon_full_fatjet1PrunedML2L3.pdf}
\end{frame}

\begin{frame}
   \frametitle{\small 161028/photon\_full\_fatjet1PrunedML2L3}
   \centering
   \includegraphics[width=0.7\linewidth]{161028/photon_full_fatjet1PrunedML2L3.pdf}
\end{frame}

\begin{frame}
   \frametitle{\small 160802/dilep\_full\_fatjet1PrunedML2L3}
   \centering
   \includegraphics[width=0.7\linewidth]{160802/dilep_full_fatjet1PrunedML2L3.pdf}
\end{frame}

\begin{frame}
   \frametitle{\small 161109\_1/full\_Zmm\_fatjet1PrunedML2L3}
   \centering
   \includegraphics[width=0.7\linewidth]{161109_1/full_Zmm_fatjet1PrunedML2L3.pdf}
\end{frame}

\begin{frame}
   \frametitle{\small 161109\_1/tight\_Zmm\_fatjet1PrunedML2L3}
   \centering
   \includegraphics[width=0.7\linewidth]{161109_1/tight_Zmm_fatjet1PrunedML2L3.pdf}
\end{frame}

\begin{frame}
   \frametitle{\small 161109\_1/nocut\_Zmm\_fatjet1PrunedML2L3}
   \centering
   \includegraphics[width=0.7\linewidth]{161109_1/nocut_Zmm_fatjet1PrunedML2L3.pdf}
\end{frame}

\begin{frame}
   \frametitle{\small 161109\_1/full\_Wen\_fatjet1PrunedML2L3}
   \centering
   \includegraphics[width=0.7\linewidth]{161109_1/full_Wen_fatjet1PrunedML2L3.pdf}
\end{frame}

\begin{frame}
   \frametitle{\small 161109\_1/tight\_Wen\_fatjet1PrunedML2L3}
   \centering
   \includegraphics[width=0.7\linewidth]{161109_1/tight_Wen_fatjet1PrunedML2L3.pdf}
\end{frame}

\begin{frame}
   \frametitle{\small 161109\_1/nocut\_Wen\_fatjet1PrunedML2L3}
   \centering
   \includegraphics[width=0.7\linewidth]{161109_1/nocut_Wen_fatjet1PrunedML2L3.pdf}
\end{frame}

\begin{frame}
   \frametitle{\small 161109\_1/full\_Wmn\_fatjet1PrunedML2L3}
   \centering
   \includegraphics[width=0.7\linewidth]{161109_1/full_Wmn_fatjet1PrunedML2L3.pdf}
\end{frame}

\begin{frame}
   \frametitle{\small 161109\_1/tight\_Wmn\_fatjet1PrunedML2L3}
   \centering
   \includegraphics[width=0.7\linewidth]{161109_1/tight_Wmn_fatjet1PrunedML2L3.pdf}
\end{frame}

\begin{frame}
   \frametitle{\small 161109\_1/nocut\_Wmn\_fatjet1PrunedML2L3}
   \centering
   \includegraphics[width=0.7\linewidth]{161109_1/nocut_Wmn_fatjet1PrunedML2L3.pdf}
\end{frame}

\begin{frame}
   \frametitle{\small 161026/photon\_nocut\_fatjet1PrunedML2L3}
   \centering
   \includegraphics[width=0.7\linewidth]{161026/photon_nocut_fatjet1PrunedML2L3.pdf}
\end{frame}

\begin{frame}
   \frametitle{\small 161027/photon\_nocut\_fatjet1PrunedML2L3}
   \centering
   \includegraphics[width=0.7\linewidth]{161027/photon_nocut_fatjet1PrunedML2L3.pdf}
\end{frame}

\begin{frame}
   \frametitle{\small 161028/photon\_nocut\_fatjet1PrunedML2L3}
   \centering
   \includegraphics[width=0.7\linewidth]{161028/photon_nocut_fatjet1PrunedML2L3.pdf}
\end{frame}

\begin{frame}
   \frametitle{\small 160802/dilep\_nocut\_fatjet1PrunedML2L3}
   \centering
   \includegraphics[width=0.7\linewidth]{160802/dilep_nocut_fatjet1PrunedML2L3.pdf}
\end{frame}

\begin{frame}
   \frametitle{\small 160726/semilep\_full\_0\_0\_fatjetPrunedML2L3}
   \centering
   \includegraphics[width=0.7\linewidth]{160726/semilep_full_0_0_fatjetPrunedML2L3.pdf}
\end{frame}

\begin{frame}
   \frametitle{\small 160726/semilep\_full\_massp\_tau21\_fatjetPrunedML2L3}
   \centering
   \includegraphics[width=0.7\linewidth]{160726/semilep_full_massp_tau21_fatjetPrunedML2L3.pdf}
\end{frame}

\begin{frame}
   \frametitle{\small 160726\_background/semilep\_full\_massp\_tau21\_fatjetPrunedML2L3}
   \centering
   \includegraphics[width=0.7\linewidth]{160726_background/semilep_full_massp_tau21_fatjetPrunedML2L3.pdf}
\end{frame}

\begin{frame}
   \frametitle{\small 160726\_midbackground/semilep\_full\_massp\_tau21\_fatjetPrunedML2L3}
   \centering
   \includegraphics[width=0.7\linewidth]{160726_midbackground/semilep_full_massp_tau21_fatjetPrunedML2L3.pdf}
\end{frame}

\begin{frame}
   \frametitle{\small 160726\_morebackground/semilep\_full\_massp\_tau21\_fatjetPrunedML2L3}
   \centering
   \includegraphics[width=0.7\linewidth]{160726_morebackground/semilep_full_massp_tau21_fatjetPrunedML2L3.pdf}
\end{frame}

\begin{frame}
   \frametitle{\small 160726/semilep\_full\_0\_1\_fatjetPrunedML2L3}
   \centering
   \includegraphics[width=0.7\linewidth]{160726/semilep_full_0_1_fatjetPrunedML2L3.pdf}
\end{frame}

\begin{frame}
   \frametitle{\small 160726/semilep\_full\_0\_3\_fatjetPrunedML2L3}
   \centering
   \includegraphics[width=0.7\linewidth]{160726/semilep_full_0_3_fatjetPrunedML2L3.pdf}
\end{frame}

\begin{frame}
   \frametitle{\small 160726\_background/semilep\_full\_0\_3\_fatjetPrunedML2L3}
   \centering
   \includegraphics[width=0.7\linewidth]{160726_background/semilep_full_0_3_fatjetPrunedML2L3.pdf}
\end{frame}

\begin{frame}
   \frametitle{\small 160726/semilep\_full\_0\_5\_fatjetPrunedML2L3}
   \centering
   \includegraphics[width=0.7\linewidth]{160726/semilep_full_0_5_fatjetPrunedML2L3.pdf}
\end{frame}

\begin{frame}
   \frametitle{\small 160726\_background/semilep\_full\_0\_5\_fatjetPrunedML2L3}
   \centering
   \includegraphics[width=0.7\linewidth]{160726_background/semilep_full_0_5_fatjetPrunedML2L3.pdf}
\end{frame}

\begin{frame}
   \frametitle{\small 160726/semilep\_full\_0\_7\_fatjetPrunedML2L3}
   \centering
   \includegraphics[width=0.7\linewidth]{160726/semilep_full_0_7_fatjetPrunedML2L3.pdf}
\end{frame}

\begin{frame}
   \frametitle{\small 160726\_background/semilep\_full\_fatjetPrunedML2L3}
   \centering
   \includegraphics[width=0.7\linewidth]{160726_background/semilep_full_fatjetPrunedML2L3.pdf}
\end{frame}

\begin{frame}
   \frametitle{\small 160726\_midbackground/semilep\_full\_fatjetPrunedML2L3}
   \centering
   \includegraphics[width=0.7\linewidth]{160726_midbackground/semilep_full_fatjetPrunedML2L3.pdf}
\end{frame}

\begin{frame}
   \frametitle{\small 160726\_morebackground/semilep\_full\_fatjetPrunedML2L3}
   \centering
   \includegraphics[width=0.7\linewidth]{160726_morebackground/semilep_full_fatjetPrunedML2L3.pdf}
\end{frame}

\begin{frame}
   \frametitle{\small 160727\_down/semilep\_full\_fatjetPrunedML2L3}
   \centering
   \includegraphics[width=0.7\linewidth]{160727_down/semilep_full_fatjetPrunedML2L3.pdf}
\end{frame}

\begin{frame}
   \frametitle{\small 160726\_up/semilep\_full\_fatjetPrunedML2L3}
   \centering
   \includegraphics[width=0.7\linewidth]{160726_up/semilep_full_fatjetPrunedML2L3.pdf}
\end{frame}

\begin{frame}
   \frametitle{\small 160726/semilep\_nocut\_nsmalljets\_fatjetPrunedML2L3}
   \centering
   \includegraphics[width=0.7\linewidth]{160726/semilep_nocut_nsmalljets_fatjetPrunedML2L3.pdf}
\end{frame}

\begin{frame}
   \frametitle{\small 160726/semilep\_full\_highpt\_fatjetPrunedML2L3}
   \centering
   \includegraphics[width=0.7\linewidth]{160726/semilep_full_highpt_fatjetPrunedML2L3.pdf}
\end{frame}

\begin{frame}
   \frametitle{\small 160726/semilep\_full\_massp\_tau21\_fatjetDRLooseB}
   \centering
   \includegraphics[width=0.7\linewidth]{160726/semilep_full_massp_tau21_fatjetDRLooseB.pdf}
\end{frame}

\begin{frame}
   \frametitle{\small 160726/semilep\_full\_fatjetDRLooseB}
   \centering
   \includegraphics[width=0.7\linewidth]{160726/semilep_full_fatjetDRLooseB.pdf}
\end{frame}

\begin{frame}
   \frametitle{\small 160726/semilep\_nocut\_nsmalljets\_fatjetDRLooseB}
   \centering
   \includegraphics[width=0.7\linewidth]{160726/semilep_nocut_nsmalljets_fatjetDRLooseB.pdf}
\end{frame}

\begin{frame}
   \frametitle{\small 160726/semilep\_full\_highpt\_fatjetDRLooseB}
   \centering
   \includegraphics[width=0.7\linewidth]{160726/semilep_full_highpt_fatjetDRLooseB.pdf}
\end{frame}

\begin{frame}
   \frametitle{\small 160726/semilep\_nocut\_fatjetDRLooseB}
   \centering
   \includegraphics[width=0.7\linewidth]{160726/semilep_nocut_fatjetDRLooseB.pdf}
\end{frame}

\begin{frame}
   \frametitle{\small met\_SR}
   \centering
   \includegraphics[width=0.7\linewidth]{met_SR.pdf}
\end{frame}

\begin{frame}
   \frametitle{\small 161026/photon\_full\_recoil}
   \centering
   \includegraphics[width=0.7\linewidth]{161026/photon_full_recoil.pdf}
\end{frame}

\begin{frame}
   \frametitle{\small 161027/photon\_full\_recoil}
   \centering
   \includegraphics[width=0.7\linewidth]{161027/photon_full_recoil.pdf}
\end{frame}

\begin{frame}
   \frametitle{\small 161028/photon\_full\_recoil}
   \centering
   \includegraphics[width=0.7\linewidth]{161028/photon_full_recoil.pdf}
\end{frame}

\begin{frame}
   \frametitle{\small 161026/photon\_nocut\_recoil}
   \centering
   \includegraphics[width=0.7\linewidth]{161026/photon_nocut_recoil.pdf}
\end{frame}

\begin{frame}
   \frametitle{\small 161027/photon\_nocut\_recoil}
   \centering
   \includegraphics[width=0.7\linewidth]{161027/photon_nocut_recoil.pdf}
\end{frame}

\begin{frame}
   \frametitle{\small 161028/photon\_nocut\_recoil}
   \centering
   \includegraphics[width=0.7\linewidth]{161028/photon_nocut_recoil.pdf}
\end{frame}

\begin{frame}
   \frametitle{\small 160726/semilep\_full\_massp\_tau21\_n\_bjetsMedium}
   \centering
   \includegraphics[width=0.7\linewidth]{160726/semilep_full_massp_tau21_n_bjetsMedium.pdf}
\end{frame}

\begin{frame}
   \frametitle{\small 160726/semilep\_full\_n\_bjetsMedium}
   \centering
   \includegraphics[width=0.7\linewidth]{160726/semilep_full_n_bjetsMedium.pdf}
\end{frame}

\begin{frame}
   \frametitle{\small 160726/semilep\_full\_highpt\_n\_bjetsMedium}
   \centering
   \includegraphics[width=0.7\linewidth]{160726/semilep_full_highpt_n_bjetsMedium.pdf}
\end{frame}

\begin{frame}
   \frametitle{\small 160726/semilep\_nocut\_n\_bjetsMedium}
   \centering
   \includegraphics[width=0.7\linewidth]{160726/semilep_nocut_n_bjetsMedium.pdf}
\end{frame}

\begin{frame}
   \frametitle{\small ttselection}
   \centering
   \includegraphics[width=0.7\linewidth]{ttselection.pdf}
\end{frame}

\begin{frame}
   \frametitle{\small 160726/semilep\_full\_massp\_tau21\_fatjetMass}
   \centering
   \includegraphics[width=0.7\linewidth]{160726/semilep_full_massp_tau21_fatjetMass.pdf}
\end{frame}

\begin{frame}
   \frametitle{\small 160726/semilep\_full\_fatjetMass}
   \centering
   \includegraphics[width=0.7\linewidth]{160726/semilep_full_fatjetMass.pdf}
\end{frame}

\begin{frame}
   \frametitle{\small 160726/semilep\_full\_highpt\_fatjetMass}
   \centering
   \includegraphics[width=0.7\linewidth]{160726/semilep_full_highpt_fatjetMass.pdf}
\end{frame}

\begin{frame}
   \frametitle{\small 160726\_background/smearedup\_mass}
   \centering
   \includegraphics[width=0.7\linewidth]{160726_background/smearedup_mass.pdf}
\end{frame}

\begin{frame}
   \frametitle{\small 161026/photon\_full\_fatjet1Pt}
   \centering
   \includegraphics[width=0.7\linewidth]{161026/photon_full_fatjet1Pt.pdf}
\end{frame}

\begin{frame}
   \frametitle{\small 161027/photon\_full\_fatjet1Pt}
   \centering
   \includegraphics[width=0.7\linewidth]{161027/photon_full_fatjet1Pt.pdf}
\end{frame}

\begin{frame}
   \frametitle{\small 161026/photon\_nocut\_fatjet1Pt}
   \centering
   \includegraphics[width=0.7\linewidth]{161026/photon_nocut_fatjet1Pt.pdf}
\end{frame}

\begin{frame}
   \frametitle{\small 161027/photon\_nocut\_fatjet1Pt}
   \centering
   \includegraphics[width=0.7\linewidth]{161027/photon_nocut_fatjet1Pt.pdf}
\end{frame}

\begin{frame}
   \frametitle{\small 160726/semilep\_full\_massp\_tau21\_fatjetPt}
   \centering
   \includegraphics[width=0.7\linewidth]{160726/semilep_full_massp_tau21_fatjetPt.pdf}
\end{frame}

\begin{frame}
   \frametitle{\small 160726/semilep\_full\_fatjetPt}
   \centering
   \includegraphics[width=0.7\linewidth]{160726/semilep_full_fatjetPt.pdf}
\end{frame}

\begin{frame}
   \frametitle{\small 160726/semilep\_full\_massp\_tau21\_n\_jetsNotFat}
   \centering
   \includegraphics[width=0.7\linewidth]{160726/semilep_full_massp_tau21_n_jetsNotFat.pdf}
\end{frame}

\begin{frame}
   \frametitle{\small 160726/semilep\_full\_n\_jetsNotFat}
   \centering
   \includegraphics[width=0.7\linewidth]{160726/semilep_full_n_jetsNotFat.pdf}
\end{frame}

\begin{frame}
   \frametitle{\small 160726/semilep\_full\_highpt\_n\_jetsNotFat}
   \centering
   \includegraphics[width=0.7\linewidth]{160726/semilep_full_highpt_n_jetsNotFat.pdf}
\end{frame}

\begin{frame}
   \frametitle{\small 160726/semilep\_nocut\_n\_jetsNotFat}
   \centering
   \includegraphics[width=0.7\linewidth]{160726/semilep_nocut_n_jetsNotFat.pdf}
\end{frame}

\begin{frame}
   \frametitle{\small 160726/semilep\_full\_massp\_tau21\_met}
   \centering
   \includegraphics[width=0.7\linewidth]{160726/semilep_full_massp_tau21_met.pdf}
\end{frame}

\begin{frame}
   \frametitle{\small 161109\_1/full\_Zee\_met}
   \centering
   \includegraphics[width=0.7\linewidth]{161109_1/full_Zee_met.pdf}
\end{frame}

\begin{frame}
   \frametitle{\small 161109\_1/tight\_Zee\_met}
   \centering
   \includegraphics[width=0.7\linewidth]{161109_1/tight_Zee_met.pdf}
\end{frame}

\begin{frame}
   \frametitle{\small 161109\_1/nocut\_Zee\_met}
   \centering
   \includegraphics[width=0.7\linewidth]{161109_1/nocut_Zee_met.pdf}
\end{frame}

\begin{frame}
   \frametitle{\small 161109\_1/full\_signal\_met}
   \centering
   \includegraphics[width=0.7\linewidth]{161109_1/full_signal_met.pdf}
\end{frame}

\begin{frame}
   \frametitle{\small 161109\_1/tight\_signal\_met}
   \centering
   \includegraphics[width=0.7\linewidth]{161109_1/tight_signal_met.pdf}
\end{frame}

\begin{frame}
   \frametitle{\small 161109\_1/nocut\_signal\_met}
   \centering
   \includegraphics[width=0.7\linewidth]{161109_1/nocut_signal_met.pdf}
\end{frame}

\begin{frame}
   \frametitle{\small 160726/semilep\_full\_met}
   \centering
   \includegraphics[width=0.7\linewidth]{160726/semilep_full_met.pdf}
\end{frame}

\begin{frame}
   \frametitle{\small 161109\_1/full\_Zmm\_met}
   \centering
   \includegraphics[width=0.7\linewidth]{161109_1/full_Zmm_met.pdf}
\end{frame}

\begin{frame}
   \frametitle{\small 161109\_1/tight\_Zmm\_met}
   \centering
   \includegraphics[width=0.7\linewidth]{161109_1/tight_Zmm_met.pdf}
\end{frame}

\begin{frame}
   \frametitle{\small 161109\_1/nocut\_Zmm\_met}
   \centering
   \includegraphics[width=0.7\linewidth]{161109_1/nocut_Zmm_met.pdf}
\end{frame}

\begin{frame}
   \frametitle{\small 161109\_1/full\_Wen\_met}
   \centering
   \includegraphics[width=0.7\linewidth]{161109_1/full_Wen_met.pdf}
\end{frame}

\begin{frame}
   \frametitle{\small 161109\_1/tight\_Wen\_met}
   \centering
   \includegraphics[width=0.7\linewidth]{161109_1/tight_Wen_met.pdf}
\end{frame}

\begin{frame}
   \frametitle{\small 161109\_1/nocut\_Wen\_met}
   \centering
   \includegraphics[width=0.7\linewidth]{161109_1/nocut_Wen_met.pdf}
\end{frame}

\begin{frame}
   \frametitle{\small 161109\_1/full\_Wmn\_met}
   \centering
   \includegraphics[width=0.7\linewidth]{161109_1/full_Wmn_met.pdf}
\end{frame}

\begin{frame}
   \frametitle{\small 161109\_1/tight\_Wmn\_met}
   \centering
   \includegraphics[width=0.7\linewidth]{161109_1/tight_Wmn_met.pdf}
\end{frame}

\begin{frame}
   \frametitle{\small 161109\_1/nocut\_Wmn\_met}
   \centering
   \includegraphics[width=0.7\linewidth]{161109_1/nocut_Wmn_met.pdf}
\end{frame}

\begin{frame}
   \frametitle{\small 161109\_1/full\_gjets\_met}
   \centering
   \includegraphics[width=0.7\linewidth]{161109_1/full_gjets_met.pdf}
\end{frame}

\begin{frame}
   \frametitle{\small 161109\_1/tight\_gjets\_met}
   \centering
   \includegraphics[width=0.7\linewidth]{161109_1/tight_gjets_met.pdf}
\end{frame}

\begin{frame}
   \frametitle{\small 161109\_1/nocut\_gjets\_met}
   \centering
   \includegraphics[width=0.7\linewidth]{161109_1/nocut_gjets_met.pdf}
\end{frame}

\begin{frame}
   \frametitle{\small 160726/semilep\_nocut\_nsmalljets\_met}
   \centering
   \includegraphics[width=0.7\linewidth]{160726/semilep_nocut_nsmalljets_met.pdf}
\end{frame}

\begin{frame}
   \frametitle{\small 160726/semilep\_full\_highpt\_met}
   \centering
   \includegraphics[width=0.7\linewidth]{160726/semilep_full_highpt_met.pdf}
\end{frame}

\begin{frame}
   \frametitle{\small 160726/semilep\_full\_massp\_tau21\_n\_bjetsTight}
   \centering
   \includegraphics[width=0.7\linewidth]{160726/semilep_full_massp_tau21_n_bjetsTight.pdf}
\end{frame}

\begin{frame}
   \frametitle{\small 160726/semilep\_full\_n\_bjetsTight}
   \centering
   \includegraphics[width=0.7\linewidth]{160726/semilep_full_n_bjetsTight.pdf}
\end{frame}

\begin{frame}
   \frametitle{\small 160726/semilep\_full\_highpt\_n\_bjetsTight}
   \centering
   \includegraphics[width=0.7\linewidth]{160726/semilep_full_highpt_n_bjetsTight.pdf}
\end{frame}

\begin{frame}
   \frametitle{\small 160707/WPt\_comparisonx}
   \centering
   \includegraphics[width=0.7\linewidth]{160707/WPt_comparisonx.pdf}
\end{frame}



\backupend

\end{document}
