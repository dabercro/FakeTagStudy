\documentclass{beamer}

\author[D. Abercrombie]{
  Daniel Abercrombie
}

\title{\bf \sffamily V-Tagging Fake Rate Scale Factor}
\date{\today}

\usecolortheme{dove}

\usepackage[absolute,overlay]{textpos}
\usefonttheme{serif}
\usepackage{appendixnumberbeamer}
\usepackage{isotope}
\usepackage{hyperref}
\usepackage[english]{babel}
\usepackage{amsmath}
\setbeamerfont{frametitle}{size=\Large,series=\bf\sffamily}
\setbeamertemplate{frametitle}[default][center]
\usepackage{siunitx}
\usepackage{tabularx}
\usepackage{makecell}

\setbeamertemplate{navigation symbols}{}
\usepackage{graphicx}
\usepackage{color}
\setbeamertemplate{footline}[text line]{\parbox{1.083\linewidth}{\footnotesize \hfill \insertshortauthor \hfill \insertpagenumber /\inserttotalframenumber}}
\setbeamertemplate{headline}[text line]{\parbox{1.083\linewidth}{\footnotesize \hspace{-0.083\linewidth} \textcolor{blue}{\sffamily \insertsection \hfill \insertsubsection}}}

\logo{\includegraphics[height=0.5cm]{/Users/dabercro/GradSchool/Presentations/MIT-logo.pdf}}

\usepackage{changepage}

\newcommand{\beginbackup}{
  \newcounter{framenumbervorappendix}
  \setcounter{framenumbervorappendix}{\value{framenumber}}
}
\newcommand{\backupend}{
  \addtocounter{framenumbervorappendix}{-\value{framenumber}}
  \addtocounter{framenumber}{\value{framenumbervorappendix}}
}

\graphicspath{{figs/}}

\begin{document}

\begin{frame}[nonumbering]
  \titlepage
\end{frame}

\begin{frame}
  \frametitle{Introduction}

  A $\gamma$ + jets selection is used to simulate background jets for V-tagging purposes.
  The scale factor can either be applied to the $\gamma$ + jets
  Mono-V control region in the global fit,
  or it can be used as a cross fit for the scale factor ``measured'' by the global fit.

  \vspace{24pt}

  A side effect of this study is showing the $\gamma$ + hadronic V
  contribution is negligible through private MadGraph samples.

\end{frame}


\begin{frame}
  \frametitle{Introduction (continued)}

  Currently, the Mono-V analysis has a measured scale factor for the signal-like backgrounds
  since these backgrounds do not have a corresponding control region.
  There is no direct scale factor for other backgrounds.
  The most significant background contributions are adjusted by a global fit.
  By using a direct scale factor, the fit complexity can be reduced.
  This scale factor can also be used for other minor backgrounds if we assume their
  tagging variable distributions (pruned mass and $\tau_2/\tau_1$)
  are shaped and mis-modelled similarly to $\gamma$ + jets.

\end{frame}

\begin{frame}
  \frametitle{Datasets}
  {\scriptsize
    \begin{itemize}
    \item GJets\_HT-*\_TuneCUETP8M1\_13TeV-madgraphMLM-pythia8
    \item QCD\_HT*\_TuneCUETP8M1\_13TeV-madgraphMLM-pythia8
    \item WJetsToLNu\_Pt-*\_TuneCUETP8M1\_13TeV-amcatnloFXFX-pythia8
    \item Privately produced $\gamma$ + hadronic vector. \\
      (Not significant, less than 0.1\% after V-tagging cuts)
    \item \textcolor{gray}{DYJetsToLL\_M-50\_HT-*\_TuneCUETP8M1\_13TeV-madgraphMLM-pythia8}
    \item \textcolor{gray}{TT\_TuneCUETP8M1\_13TeV-powheg-pythia8}
    \item \textcolor{gray}{ST\_t\{,W\}-*\_inclusiveDecays\_13TeV-powhegV2-madspin-pythia8\_TuneCUETP8M1}
    \item \textcolor{gray}{\{WW,WZ,ZZ\}\_TuneCUETP8M1\_13TeV-pythia8}
    \end{itemize}
  }
\end{frame}

\begin{frame}
  \frametitle{Selection}
  This is almost the same as the mono-jet selection except
  there is no veto on V jets and no noise cleaning cuts \\
  (like mono-jet ID and $\Delta\phi$ between jets and recoil)
  \begin{itemize}
  \item Veto loose b-jets (csv $> 0.605$)
  \item Recoil $> \SI{250}{GeV}$
  \item Fat jet $p_T > \SI{250}{GeV}$
  \item Veto $\tau$
  \item One medium photon ($p_T > \SI{175}{GeV}$)
  \item Veto leptons or additional $\gamma$
  \end{itemize}
\end{frame}

\begin{frame}
  \frametitle{$\gamma$ + jets Distributions}
  The V-tagging variables before V-tagging selections
  \begin{columns}
    \begin{column}{0.5\linewidth}
      \centering
      \includegraphics[width=\linewidth]{161027/photon_nocut_fatjet1PrunedML2L3.pdf}
    \end{column}
    \begin{column}{0.5\linewidth}
      \centering
      \includegraphics[width=\linewidth]{161027/photon_nocut_fatjet1tau21.pdf}
    \end{column}
  \end{columns}
  My $\gamma$ + jets selection seems to have a good yeild agreement.
\end{frame}

\begin{frame}
  \frametitle{$\gamma$ + jets Distributions After Tagging}
  The V-tagging variables after selections \\
  ($\tau_2/\tau_1 < 0.6$ and $\SI{65}{GeV} < m_{pruned} < \SI{105}{GeV}$)
  \begin{columns}
    \begin{column}{0.5\linewidth}
      \centering
      \includegraphics[width=\linewidth]{161027/photon_full_fatjet1PrunedML2L3.pdf}
    \end{column}
    \begin{column}{0.5\linewidth}
      \centering
      \includegraphics[width=\linewidth]{161027/photon_full_fatjet1tau21.pdf}
    \end{column}
  \end{columns}
  Yield is underestimated slightly, so scale factor less than one expected.
\end{frame}

\begin{frame}
  \frametitle{Method and Systematics}
  We repeat the method of getting the V-tagging scale factor with a cut and count analysis.
  (\href{https://indico.cern.ch/event/559594/contributions/2257923/attachments/1316800/1973048/dabercro_WTagStudy_160727.pdf}
  {Details here})

  Systematics from the W-Tag study:
  \begin{itemize}
  \item Jet Smearing -- hybrid smearing from [*] affects the mass and $p_T$ of the jet
  \item \textcolor{gray}{Background subtraction magnitude}
  \item \textcolor{gray}{Which background is subtracted}
  \item Shower effect on purity cuts -- instead, we'll compare QCD and photon purity modelling
  \end{itemize}

  \textcolor{gray}{
    The ``backgrounds'' are not relevant.
    $\gamma$ + hadronic V is actually the largest contributing signal-like background in this region.
  }

  {\small [*] 
    \href{https://twiki.cern.ch/twiki/bin/viewauth/CMS/JetResolution#Smearing_procedures}
         {https://twiki.cern.ch/twiki/bin/viewauth/CMS/JetResolution}}
\end{frame}

\begin{frame}
  \frametitle{$\gamma$ + jets Scale Factors}

  \begin{adjustwidth}{-1.5em}{-1.5em}
    {\tiny
      \begin{tabular}{c|c|c|c|c}
        \hline
        & No Cut & $m_\text{pruned}$ & $\tau_2/\tau_1$ & V-tag cut \\
        \hline
        \makecell{Background \\ Subtracted \\ Data} & 146220.46 $\pm$ 382.63 & 17155.45 $\pm$ 131.34 & 33067.40 $\pm$ 182.18 & 12468.01 $\pm$ 112.07 \\
        \makecell{Signal-\\ matched MC} & 148741.23 $\pm$ 663.36 & 19194.63 $\pm$ 233.88 & 36429.22 $\pm$ 320.87 & 13401.97 $\pm$ 199.18 \\
        \hline
        \makecell{Normalized \\ Ratio} & 1.00 $\pm$ 0.01 & 0.91 $\pm$ 0.01 & 0.92 $\pm$ 0.01 & \fcolorbox{red}{yellow}{0.95 $\pm$ 0.02} \\
        \hline
      \end{tabular}
    }
  \end{adjustwidth}

  \vspace{24pt}

  Background subtraction is the signal-like backgrounds: top, di-boson, $\gamma$ + V.
  Together, these are a fraction of a percent. 

\end{frame}

\begin{frame}
  \frametitle{Systematic Uncertianties}

  QCD gives $\pm 0.3$ \\

  \begin{adjustwidth}{-1.5em}{-1.5em}
    {\tiny
      \begin{tabular}{c|c|c|c|c}
        \hline
        & No Cut & $m_\text{pruned}$ & $\tau_2/\tau_1$ & V-tag cut \\
        \hline
        \makecell{Background \\ Subtracted \\ Data} & 146230.21 $\pm$ 382.64 & 17155.76 $\pm$ 131.33 & 33066.12 $\pm$ 182.17 & 12466.24 $\pm$ 112.05 \\
        \makecell{Signal-\\ matched MC} & 152036.24 $\pm$ 735.88 & 20334.27 $\pm$ 278.48 & 37959.01 $\pm$ 367.05 & 14086.83 $\pm$ 234.96 \\
        \hline
        \makecell{Normalized \\ Ratio} & 1.00 $\pm$ 0.01 & 0.88 $\pm$ 0.01 & 0.91 $\pm$ 0.01 & \fcolorbox{red}{yellow}{0.92 $\pm$ 0.02} \\
        \hline
      \end{tabular}
    }
  \end{adjustwidth}

  \vspace{12pt}
  Smearing (up gives the largest difference) gives $\pm 0.02$ \\

  \begin{adjustwidth}{-1.5em}{-1.5em}
    {\tiny
      \begin{tabular}{c|c|c|c|c}
        \hline
        & No Cut & $m_\text{pruned}$ & $\tau_2/\tau_1$ & V-tag cut \\
        \hline
        \makecell{Background \\ Subtracted \\ Data} & 199613.57 $\pm$ 447.07 & 17154.75 $\pm$ 131.34 & 33067.05 $\pm$ 182.18 & 12468.81 $\pm$ 112.07 \\
        \makecell{Signal-\\ matched MC} & 204154.34 $\pm$ 823.25 & 19872.40 $\pm$ 236.32 & 37113.52 $\pm$ 323.20 & 13135.39 $\pm$ 193.35 \\
        \hline
        \makecell{Normalized \\ Ratio} & 1.00 $\pm$ 0.00 & 0.88 $\pm$ 0.01 & 0.91 $\pm$ 0.01 & \fcolorbox{red}{yellow}{0.97 $\pm$ 0.02} \\
        \hline
      \end{tabular}
    }
  \end{adjustwidth}

\end{frame}

\begin{frame}
  \frametitle{Recoil Dependence?}
  With $\gamma$ + jets, we may be able to measure any recoil dependence on the fake scale factor
  \begin{columns}
    \begin{column}{0.5\linewidth}
      \centering
      Before V-tagging \\
      \includegraphics[width=\linewidth]{161027/photon_nocut_recoil.pdf}
    \end{column}
    \begin{column}{0.5\linewidth}
      After V-tagging \\
      \includegraphics[width=\linewidth]{161027/photon_full_recoil.pdf}
    \end{column}
  \end{columns}
\end{frame}

\begin{frame}
  \frametitle{Conclusions}

  \begin{itemize}
  \item At a first pass, I get a scale factor of \fcolorbox{red}{yellow}{$0.95 \pm 0.04$}.
  \item This can be applied to the $\gamma$ + jets Mono-V control region
    to see how the fit is affected.
  \item Recoil dependence can also be determined, if desired
  \end{itemize}

\end{frame}

\beginbackup

\begin{frame}
  \frametitle{Backup Slides}
\end{frame}

\begin{frame}
   \frametitle{\small 160803/dimu\_full\_fatjet1DRGenW}
   \centering
   \includegraphics[width=0.7\linewidth]{160803/dimu_full_fatjet1DRGenW.pdf}
\end{frame}

\begin{frame}
   \frametitle{\small 160803/dimu\_nocut\_fatjet1DRGenW}
   \centering
   \includegraphics[width=0.7\linewidth]{160803/dimu_nocut_fatjet1DRGenW.pdf}
\end{frame}

\begin{frame}
   \frametitle{\small 160803/photon\_full\_recoil}
   \centering
   \includegraphics[width=0.7\linewidth]{160803/photon_full_recoil.pdf}
\end{frame}

\begin{frame}
   \frametitle{\small 160803/dimu\_full\_recoil}
   \centering
   \includegraphics[width=0.7\linewidth]{160803/dimu_full_recoil.pdf}
\end{frame}

\begin{frame}
   \frametitle{\small 160803/photon\_nocut\_recoil}
   \centering
   \includegraphics[width=0.7\linewidth]{160803/photon_nocut_recoil.pdf}
\end{frame}

\begin{frame}
   \frametitle{\small 160803/dimu\_nocut\_recoil}
   \centering
   \includegraphics[width=0.7\linewidth]{160803/dimu_nocut_recoil.pdf}
\end{frame}

\begin{frame}
   \frametitle{\small 160803/dimu\_full\_dilepMass}
   \centering
   \includegraphics[width=0.7\linewidth]{160803/dimu_full_dilepMass.pdf}
\end{frame}

\begin{frame}
   \frametitle{\small 160803/dimu\_nocut\_dilepMass}
   \centering
   \includegraphics[width=0.7\linewidth]{160803/dimu_nocut_dilepMass.pdf}
\end{frame}

\begin{frame}
   \frametitle{\small 160803/dimu\_full\_fatjet1Pt}
   \centering
   \includegraphics[width=0.7\linewidth]{160803/dimu_full_fatjet1Pt.pdf}
\end{frame}

\begin{frame}
   \frametitle{\small 160803/dimu\_nocut\_fatjet1Pt}
   \centering
   \includegraphics[width=0.7\linewidth]{160803/dimu_nocut_fatjet1Pt.pdf}
\end{frame}



\backupend

\end{document}
