\documentclass{beamer}

\author[D. Abercrombie]{
  Daniel Abercrombie
}

\title{\bf \sffamily Title Here}
\date{\today}

\usecolortheme{dove}

\usepackage[absolute,overlay]{textpos}
\usefonttheme{serif}
\usepackage{appendixnumberbeamer}
\usepackage{isotope}
\usepackage{hyperref}
\usepackage[english]{babel}
\usepackage{amsmath}
\setbeamerfont{frametitle}{size=\Large,series=\bf\sffamily}
\setbeamertemplate{frametitle}[default][center]
\usepackage{siunitx}
\usepackage{tabularx}
\usepackage{makecell}

\setbeamertemplate{navigation symbols}{}
\usepackage{graphicx}
\usepackage{color}
\setbeamertemplate{footline}[text line]{\parbox{1.083\linewidth}{\footnotesize \hfill \insertshortauthor \hfill \insertpagenumber /\inserttotalframenumber}}
\setbeamertemplate{headline}[text line]{\parbox{1.083\linewidth}{\footnotesize \hspace{-0.083\linewidth} \textcolor{blue}{\sffamily \insertsection \hfill \insertsubsection}}}

\usepackage{changepage}

\newcommand{\beginbackup}{
  \newcounter{framenumbervorappendix}
  \setcounter{framenumbervorappendix}{\value{framenumber}}
}
\newcommand{\backupend}{
  \addtocounter{framenumbervorappendix}{-\value{framenumber}}
  \addtocounter{framenumber}{\value{framenumbervorappendix}} 
}

\graphicspath{{figs/}}

\begin{document}

\begin{frame}[nonumbering]
  \titlepage
\end{frame}

\begin{frame}
  \frametitle{Introduction}

  The Hadronic Mono-V analysis is an analysis orthogonal to mono-jet,
  with particularly good sensitivity for scalar mediators
  (as I understand it).

\end{frame}

\begin{frame}
  \frametitle{Currently used V-tagging variables}

  \begin{columns}
    \begin{column}{0.5\linewidth}
      \includegraphics[width=0.5\linewidth]{161109_1/nocut_signal_fatjet1tau21.pdf} \\
      \includegraphics[width=0.5\linewidth]{161109_1/nocut_gjets_fatjet1tau21.pdf}
    \end{column}
    \begin{column}{0.5\linewidth}
      \includegraphics[width=0.5\linewidth]{161109_1/nocut_signal_fatjet1PrunedML2L3.pdf} \\
      \includegraphics[width=0.5\linewidth]{161109_1/nocut_gjets_fatjet1PrunedML2L3.pdf}
    \end{column}
  \end{columns}

\end{frame}

\begin{frame}
  \frametitle{Potentially tighter cuts?}

  \begin{columns}
    \begin{column}{0.5\linewidth}
      Current \\
      \includegraphics[width=0.5\linewidth]{161109_1/full_signal_fatjet1PrunedML2L3.pdf} \\
      \includegraphics[width=0.5\linewidth]{161109_1/full_gjets_fatjet1PrunedML2L3.pdf}
    \end{column}
    \begin{column}{0.5\linewidth}
      Tight \\
      \includegraphics[width=0.5\linewidth]{161109_1/tight_signal_fatjet1PrunedML2L3.pdf} \\
      \includegraphics[width=0.5\linewidth]{161109_1/tight_gjets_fatjet1PrunedML2L3.pdf}
    \end{column}
  \end{columns}

  I don't like how they look in $\gamma$ + jets, a rather important region for the analysis

\end{frame}

\begin{frame}
  \frametitle{Scale Factors: Signal}
  Can simplify or help understand the global fit

  \vspace{12pt}
  \begin{columns}
    \begin{column}{0.5\linewidth}
      \centering
      \textcolor{blue}{Pruned Mass}
      \includegraphics[width=\linewidth]
                      {160726/semilep_full_fatjetPrunedML2L3.pdf}
    \end{column}
    \begin{column}{0.5\linewidth}
      \centering
      \textcolor{blue}{N-subjettiness ($\tau_2/\tau_1$)}
      \includegraphics[width=\linewidth]
                      {160726/semilep_full_fatjettau21.pdf}
    \end{column}
  \end{columns}
  Cut and count: \boxed{$0.97 \pm 0.04 \text{(stat.)} \pm 0.08 \text{(sys.)}$} \\
  Tighter $\tau_2/\tau_1$: \boxed{$0.92 \pm 0.04 \text{(stat.)} \pm 0.09 \text{(sys.)}$} \\
  Details: \url{http://t3serv001.mit.edu/~dabercro/dabercro_WTagStudy_160727.pdf}

\end{frame}

\begin{frame}

  \begin{columns}
    \begin{column}{0.5\linewidth}
      \centering
      \includegraphics[width=\linewidth]{161027/photon_full_fatjet1PrunedML2L3.pdf}
    \end{column}
    \begin{column}{0.5\linewidth}
      \centering
      \includegraphics[width=\linewidth]{161027/photon_full_fatjet1tau21.pdf}
    \end{column}
  \end{columns}

  Cut and count: \boxed{$0.95 \pm 0.02 \text{(stat.)} \pm 0.035 \text{(sys.)}$} \\
  Tighter: Glorious crash needs investigating

\end{frame}

\begin{frame}
  \frametitle{}
  \textcolor{blue}{\scriptsize \hspace{-35pt} Signal: Axial
    $m_\text{med} = \SI{2000}{GeV}$,
    $m_\text{DM} = \SI{50}{GeV}$}
  \begin{columns}
    \begin{column}{0.5\linewidth}
      \centering
      \includegraphics[width=\linewidth]{160719/fatjetPrunedMx.pdf}
    \end{column}
    \begin{column}{0.5\linewidth}
      There is some signal lost when we assume that a jet is merged.
      It would be nice to think about how to get it back.
      (I have no solutions at the moment.)
    \end{column}
  \end{columns}
\end{frame}

\begin{frame}
  \frametitle{My Opinions}

  The current monojet and mono-V analyses are redone versions of the \SI{8}{TeV} analysis.
  I don't really want to graduate on that.
  The analyses are systematically limited, and will require a significant improvement to move past that.

  Tagging boosted objects works, and is cleaner than resolved analyses, but it's relatively lazy.

\end{frame}

\beginbackup

\begin{frame}
  \frametitle{Backup Slides}
\end{frame}

\begin{frame}
   \frametitle{\small 160803/dimu\_full\_fatjet1DRGenW}
   \centering
   \includegraphics[width=0.7\linewidth]{160803/dimu_full_fatjet1DRGenW.pdf}
\end{frame}

\begin{frame}
   \frametitle{\small 160803/dimu\_nocut\_fatjet1DRGenW}
   \centering
   \includegraphics[width=0.7\linewidth]{160803/dimu_nocut_fatjet1DRGenW.pdf}
\end{frame}

\begin{frame}
   \frametitle{\small 160803/photon\_full\_recoil}
   \centering
   \includegraphics[width=0.7\linewidth]{160803/photon_full_recoil.pdf}
\end{frame}

\begin{frame}
   \frametitle{\small 160803/dimu\_full\_recoil}
   \centering
   \includegraphics[width=0.7\linewidth]{160803/dimu_full_recoil.pdf}
\end{frame}

\begin{frame}
   \frametitle{\small 160803/photon\_nocut\_recoil}
   \centering
   \includegraphics[width=0.7\linewidth]{160803/photon_nocut_recoil.pdf}
\end{frame}

\begin{frame}
   \frametitle{\small 160803/dimu\_nocut\_recoil}
   \centering
   \includegraphics[width=0.7\linewidth]{160803/dimu_nocut_recoil.pdf}
\end{frame}

\begin{frame}
   \frametitle{\small 160803/dimu\_full\_dilepMass}
   \centering
   \includegraphics[width=0.7\linewidth]{160803/dimu_full_dilepMass.pdf}
\end{frame}

\begin{frame}
   \frametitle{\small 160803/dimu\_nocut\_dilepMass}
   \centering
   \includegraphics[width=0.7\linewidth]{160803/dimu_nocut_dilepMass.pdf}
\end{frame}

\begin{frame}
   \frametitle{\small 160803/dimu\_full\_fatjet1Pt}
   \centering
   \includegraphics[width=0.7\linewidth]{160803/dimu_full_fatjet1Pt.pdf}
\end{frame}

\begin{frame}
   \frametitle{\small 160803/dimu\_nocut\_fatjet1Pt}
   \centering
   \includegraphics[width=0.7\linewidth]{160803/dimu_nocut_fatjet1Pt.pdf}
\end{frame}



\backupend

\end{document}
