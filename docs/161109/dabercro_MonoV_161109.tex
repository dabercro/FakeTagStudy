\documentclass{beamer}

\author[D. Abercrombie]{
  Daniel Abercrombie
}

\title{\bf \sffamily Some Things About Mono-V}
\date{\today}

\usecolortheme{dove}

\usepackage[absolute,overlay]{textpos}
\usefonttheme{serif}
\usepackage{appendixnumberbeamer}
\usepackage{isotope}
\usepackage{hyperref}
\usepackage[english]{babel}
\usepackage{amsmath}
\setbeamerfont{frametitle}{size=\Large,series=\bf\sffamily}
\setbeamertemplate{frametitle}[default][center]
\usepackage{siunitx}
\usepackage{tabularx}
\usepackage{makecell}

\setbeamertemplate{navigation symbols}{}
\usepackage{graphicx}
\usepackage{color}
\setbeamertemplate{footline}[text line]{\parbox{1.083\linewidth}{\footnotesize \hfill \insertshortauthor \hfill \insertpagenumber /\inserttotalframenumber}}
\setbeamertemplate{headline}[text line]{\parbox{1.083\linewidth}{\footnotesize \hspace{-0.083\linewidth} \textcolor{blue}{\sffamily \insertsection \hfill \insertsubsection}}}

\usepackage{changepage}

\newcommand{\beginbackup}{
  \newcounter{framenumbervorappendix}
  \setcounter{framenumbervorappendix}{\value{framenumber}}
}
\newcommand{\backupend}{
  \addtocounter{framenumbervorappendix}{-\value{framenumber}}
  \addtocounter{framenumber}{\value{framenumbervorappendix}} 
}

\graphicspath{{figs/}}

\begin{document}

\begin{frame}[nonumbering]
  \titlepage
\end{frame}

\begin{frame}
  \frametitle{Introduction}

  The Hadronic Mono-V analysis is an analysis orthogonal to mono-jet,
  with particularly good sensitivity for scalar mediators
  (as I understand it).

\end{frame}

\begin{frame}
  \frametitle{Currently used V-tagging variables}

  \begin{columns}
    \begin{column}{0.5\linewidth}
      \centering
      \includegraphics[width=0.6\linewidth]{161109_1/nocut_signal_fatjet1tau21.pdf} \\
      \includegraphics[width=0.6\linewidth]{161109_1/nocut_gjets_fatjet1tau21.pdf}
    \end{column}
    \begin{column}{0.5\linewidth}
      \centering
      \includegraphics[width=0.6\linewidth]{161109_1/nocut_signal_fatjet1PrunedML2L3.pdf} \\
      \includegraphics[width=0.6\linewidth]{161109_1/nocut_gjets_fatjet1PrunedML2L3.pdf}
    \end{column}
  \end{columns}

\end{frame}

\begin{frame}
  \frametitle{Potentially tighter cuts?}

  \begin{columns}
    \begin{column}{0.5\linewidth}
      \centering
      Current \\
      \includegraphics[width=0.6\linewidth]{161109_1/full_signal_fatjet1PrunedML2L3.pdf} \\
      \includegraphics[width=0.6\linewidth]{161109_1/full_gjets_fatjet1PrunedML2L3.pdf}
    \end{column}
    \begin{column}{0.5\linewidth}
      \centering
      Tight \\
      \includegraphics[width=0.6\linewidth]{161109_1/tight_signal_fatjet1PrunedML2L3.pdf} \\
      \includegraphics[width=0.6\linewidth]{161109_1/tight_gjets_fatjet1PrunedML2L3.pdf}
    \end{column}
  \end{columns}

\end{frame}

\begin{frame}
  \frametitle{Potentially tighter cuts?}

  \begin{columns}
    \begin{column}{0.5\linewidth}
      \includegraphics[width=\linewidth]{161109_1/full_gjets_met.pdf}
    \end{column}
    \begin{column}{0.5\linewidth}
      \includegraphics[width=\linewidth]{161109_1/tight_gjets_met.pdf}
    \end{column}
  \end{columns}

  I don't like how they look in $\gamma$ + jets, a rather important region for the analysis

\end{frame}

\begin{frame}
  \frametitle{Scale Factors: Signal}

  Can simplify or help understand the global fit

  \begin{columns}
    \begin{column}{0.5\linewidth}
      \centering
      \textcolor{blue}{Pruned Mass}
      \includegraphics[width=\linewidth]
                      {160726/semilep_full_fatjetPrunedML2L3.pdf}
    \end{column}
    \begin{column}{0.5\linewidth}
      \centering
      \textcolor{blue}{N-subjettiness ($\tau_2/\tau_1$)}
      \includegraphics[width=\linewidth]
                      {160726/semilep_full_fatjettau21.pdf}
    \end{column}
  \end{columns}

  Cut and count: \boxed{$0.97 \pm 0.04 \text{(stat.)} \pm 0.08 \text{(sys.)}$} \\
  Tighter $\tau_2/\tau_1$: \boxed{$0.92 \pm 0.04 \text{(stat.)} \pm 0.09 \text{(sys.)}$} \\
  Details: \url{http://t3serv001.mit.edu/~dabercro/dabercro_WTagStudy_160727.pdf}

\end{frame}

\begin{frame}
  \frametitle{Scale Factors: Background}

  \begin{columns}
    \begin{column}{0.5\linewidth}
      \centering
      \includegraphics[width=\linewidth]{161027/photon_full_fatjet1PrunedML2L3.pdf}
    \end{column}
    \begin{column}{0.5\linewidth}
      \centering
      \includegraphics[width=\linewidth]{161027/photon_full_fatjet1tau21.pdf}
    \end{column}
  \end{columns}

  Cut and count: \boxed{$0.95 \pm 0.02 \text{(stat.)} \pm 0.035 \text{(sys.)}$} \\
  Tighter: Glorious crash needs investigating

\end{frame}

\begin{frame}
  \frametitle{Numbers Changing With Pileup}

  Example: fake scale factor \\

  \vspace{24pt}
  \centering

  NPV 10 to 20: $0.94 $\pm$ 0.02$ \\
  NPV 20 to 30: $0.90 $\pm$ 0.03$ \\ 
  NPV 30 to 40: $0.86 $\pm$ 0.09$

\end{frame}

\begin{frame}
  \frametitle{Reweighing}

  \begin{columns}
    \begin{column}{0.5\linewidth}
      \centering
      \includegraphics[width=\linewidth]{161109/checkshape.pdf}
    \end{column}
    \begin{column}{0.5\linewidth}
      \centering
      \includegraphics[width=\linewidth]{161027/photon_nocut_fatjet1PrunedML2L3.pdf}
    \end{column}
  \end{columns}

\end{frame}

\begin{frame}
  \frametitle{Conclusions}

  \begin{itemize}
  \item There's not much time to change things for Moriond,
    but I expect more problems with systematics as pileup increases.
  \item With our clean method of doing scale factors,
    we can do something much messier for tagging.
  \end{itemize}

\end{frame}

\beginbackup

\begin{frame}
  \frametitle{Backup Slides}
\end{frame}

\begin{frame}
  \frametitle{W-jets Aren't Always Merged}
  \textcolor{blue}{\scriptsize \hspace{-35pt} Signal: Axial
    $m_\text{med} = \SI{2000}{GeV}$,
    $m_\text{DM} = \SI{50}{GeV}$}
  \begin{columns}
    \begin{column}{0.5\linewidth}
      \centering
      \includegraphics[width=\linewidth]{160719/fatjetPrunedMx.pdf}
    \end{column}
    \begin{column}{0.5\linewidth}
      There is some signal lost when we assume that a jet is merged.
      It would be nice to think about how to get it back.
      (I have no solutions at the moment.)
    \end{column}
  \end{columns}
\end{frame}

\begin{frame}
  \frametitle{My Opinions}

  The current monojet and mono-V analyses are redone versions of the \SI{8}{TeV} analysis.
  I don't really want to graduate on that.
  The analyses are systematically limited, and will require a significant improvement to move past that.

  Tagging boosted objects works, and is cleaner than resolved analyses, but it's relatively lazy.

\end{frame}

\begin{frame}
   \frametitle{\small 161109\_1/tight\_Zee\_fatjet1tau21}
   \centering
   \includegraphics[width=0.7\linewidth]{161109_1/tight_Zee_fatjet1tau21.pdf}
\end{frame}

\begin{frame}
   \frametitle{\small 161109\_1/nocut\_Zee\_fatjet1tau21}
   \centering
   \includegraphics[width=0.7\linewidth]{161109_1/nocut_Zee_fatjet1tau21.pdf}
\end{frame}

\begin{frame}
   \frametitle{\small 161109\_1/tight\_signal\_fatjet1tau21}
   \centering
   \includegraphics[width=0.7\linewidth]{161109_1/tight_signal_fatjet1tau21.pdf}
\end{frame}

\begin{frame}
   \frametitle{\small 161026/photon\_full\_fatjet1tau21}
   \centering
   \includegraphics[width=0.7\linewidth]{161026/photon_full_fatjet1tau21.pdf}
\end{frame}

\begin{frame}
   \frametitle{\small 161028/photon\_full\_fatjet1tau21}
   \centering
   \includegraphics[width=0.7\linewidth]{161028/photon_full_fatjet1tau21.pdf}
\end{frame}

\begin{frame}
   \frametitle{\small 160802/dilep\_full\_fatjet1tau21}
   \centering
   \includegraphics[width=0.7\linewidth]{160802/dilep_full_fatjet1tau21.pdf}
\end{frame}

\begin{frame}
   \frametitle{\small 161109\_1/tight\_Zmm\_fatjet1tau21}
   \centering
   \includegraphics[width=0.7\linewidth]{161109_1/tight_Zmm_fatjet1tau21.pdf}
\end{frame}

\begin{frame}
   \frametitle{\small 161109\_1/nocut\_Zmm\_fatjet1tau21}
   \centering
   \includegraphics[width=0.7\linewidth]{161109_1/nocut_Zmm_fatjet1tau21.pdf}
\end{frame}

\begin{frame}
   \frametitle{\small 161109\_1/tight\_Wen\_fatjet1tau21}
   \centering
   \includegraphics[width=0.7\linewidth]{161109_1/tight_Wen_fatjet1tau21.pdf}
\end{frame}

\begin{frame}
   \frametitle{\small 161109\_1/nocut\_Wen\_fatjet1tau21}
   \centering
   \includegraphics[width=0.7\linewidth]{161109_1/nocut_Wen_fatjet1tau21.pdf}
\end{frame}

\begin{frame}
   \frametitle{\small 161109\_1/tight\_Wmn\_fatjet1tau21}
   \centering
   \includegraphics[width=0.7\linewidth]{161109_1/tight_Wmn_fatjet1tau21.pdf}
\end{frame}

\begin{frame}
   \frametitle{\small 161109\_1/nocut\_Wmn\_fatjet1tau21}
   \centering
   \includegraphics[width=0.7\linewidth]{161109_1/nocut_Wmn_fatjet1tau21.pdf}
\end{frame}

\begin{frame}
   \frametitle{\small 161109\_1/tight\_gjets\_fatjet1tau21}
   \centering
   \includegraphics[width=0.7\linewidth]{161109_1/tight_gjets_fatjet1tau21.pdf}
\end{frame}

\begin{frame}
   \frametitle{\small 161026/photon\_nocut\_fatjet1tau21}
   \centering
   \includegraphics[width=0.7\linewidth]{161026/photon_nocut_fatjet1tau21.pdf}
\end{frame}

\begin{frame}
   \frametitle{\small 161027/photon\_nocut\_fatjet1tau21}
   \centering
   \includegraphics[width=0.7\linewidth]{161027/photon_nocut_fatjet1tau21.pdf}
\end{frame}

\begin{frame}
   \frametitle{\small 161028/photon\_nocut\_fatjet1tau21}
   \centering
   \includegraphics[width=0.7\linewidth]{161028/photon_nocut_fatjet1tau21.pdf}
\end{frame}

\begin{frame}
   \frametitle{\small 160802/dilep\_nocut\_fatjet1tau21}
   \centering
   \includegraphics[width=0.7\linewidth]{160802/dilep_nocut_fatjet1tau21.pdf}
\end{frame}

\begin{frame}
   \frametitle{\small 160726/semilep\_full\_massp\_tau21\_fatjettau21}
   \centering
   \includegraphics[width=0.7\linewidth]{160726/semilep_full_massp_tau21_fatjettau21.pdf}
\end{frame}

\begin{frame}
   \frametitle{\small 160726\_background/semilep\_full\_massp\_tau21\_fatjettau21}
   \centering
   \includegraphics[width=0.7\linewidth]{160726_background/semilep_full_massp_tau21_fatjettau21.pdf}
\end{frame}

\begin{frame}
   \frametitle{\small 160726\_midbackground/semilep\_full\_massp\_tau21\_fatjettau21}
   \centering
   \includegraphics[width=0.7\linewidth]{160726_midbackground/semilep_full_massp_tau21_fatjettau21.pdf}
\end{frame}

\begin{frame}
   \frametitle{\small 160726\_morebackground/semilep\_full\_massp\_tau21\_fatjettau21}
   \centering
   \includegraphics[width=0.7\linewidth]{160726_morebackground/semilep_full_massp_tau21_fatjettau21.pdf}
\end{frame}

\begin{frame}
   \frametitle{\small 160726\_background/semilep\_full\_fatjettau21}
   \centering
   \includegraphics[width=0.7\linewidth]{160726_background/semilep_full_fatjettau21.pdf}
\end{frame}

\begin{frame}
   \frametitle{\small 160726\_midbackground/semilep\_full\_fatjettau21}
   \centering
   \includegraphics[width=0.7\linewidth]{160726_midbackground/semilep_full_fatjettau21.pdf}
\end{frame}

\begin{frame}
   \frametitle{\small 160726\_morebackground/semilep\_full\_fatjettau21}
   \centering
   \includegraphics[width=0.7\linewidth]{160726_morebackground/semilep_full_fatjettau21.pdf}
\end{frame}

\begin{frame}
   \frametitle{\small 160727\_down/semilep\_full\_fatjettau21}
   \centering
   \includegraphics[width=0.7\linewidth]{160727_down/semilep_full_fatjettau21.pdf}
\end{frame}

\begin{frame}
   \frametitle{\small 160726\_up/semilep\_full\_fatjettau21}
   \centering
   \includegraphics[width=0.7\linewidth]{160726_up/semilep_full_fatjettau21.pdf}
\end{frame}

\begin{frame}
   \frametitle{\small 160726/semilep\_full\_highpt\_fatjettau21}
   \centering
   \includegraphics[width=0.7\linewidth]{160726/semilep_full_highpt_fatjettau21.pdf}
\end{frame}

\begin{frame}
   \frametitle{\small 160726/semilep\_nocut\_nsmalljets\_fatjetDPhiLep1}
   \centering
   \includegraphics[width=0.7\linewidth]{160726/semilep_nocut_nsmalljets_fatjetDPhiLep1.pdf}
\end{frame}

\begin{frame}
   \frametitle{\small 161109\_1/full\_Zee\_fatjet1PrunedML2L3}
   \centering
   \includegraphics[width=0.7\linewidth]{161109_1/full_Zee_fatjet1PrunedML2L3.pdf}
\end{frame}

\begin{frame}
   \frametitle{\small 161109\_1/tight\_Zee\_fatjet1PrunedML2L3}
   \centering
   \includegraphics[width=0.7\linewidth]{161109_1/tight_Zee_fatjet1PrunedML2L3.pdf}
\end{frame}

\begin{frame}
   \frametitle{\small 161109\_1/nocut\_Zee\_fatjet1PrunedML2L3}
   \centering
   \includegraphics[width=0.7\linewidth]{161109_1/nocut_Zee_fatjet1PrunedML2L3.pdf}
\end{frame}

\begin{frame}
   \frametitle{\small 161026/photon\_full\_fatjet1PrunedML2L3}
   \centering
   \includegraphics[width=0.7\linewidth]{161026/photon_full_fatjet1PrunedML2L3.pdf}
\end{frame}

\begin{frame}
   \frametitle{\small 161028/photon\_full\_fatjet1PrunedML2L3}
   \centering
   \includegraphics[width=0.7\linewidth]{161028/photon_full_fatjet1PrunedML2L3.pdf}
\end{frame}

\begin{frame}
   \frametitle{\small 160802/dilep\_full\_fatjet1PrunedML2L3}
   \centering
   \includegraphics[width=0.7\linewidth]{160802/dilep_full_fatjet1PrunedML2L3.pdf}
\end{frame}

\begin{frame}
   \frametitle{\small 161109\_1/full\_Zmm\_fatjet1PrunedML2L3}
   \centering
   \includegraphics[width=0.7\linewidth]{161109_1/full_Zmm_fatjet1PrunedML2L3.pdf}
\end{frame}

\begin{frame}
   \frametitle{\small 161109\_1/tight\_Zmm\_fatjet1PrunedML2L3}
   \centering
   \includegraphics[width=0.7\linewidth]{161109_1/tight_Zmm_fatjet1PrunedML2L3.pdf}
\end{frame}

\begin{frame}
   \frametitle{\small 161109\_1/nocut\_Zmm\_fatjet1PrunedML2L3}
   \centering
   \includegraphics[width=0.7\linewidth]{161109_1/nocut_Zmm_fatjet1PrunedML2L3.pdf}
\end{frame}

\begin{frame}
   \frametitle{\small 161109\_1/full\_Wen\_fatjet1PrunedML2L3}
   \centering
   \includegraphics[width=0.7\linewidth]{161109_1/full_Wen_fatjet1PrunedML2L3.pdf}
\end{frame}

\begin{frame}
   \frametitle{\small 161109\_1/tight\_Wen\_fatjet1PrunedML2L3}
   \centering
   \includegraphics[width=0.7\linewidth]{161109_1/tight_Wen_fatjet1PrunedML2L3.pdf}
\end{frame}

\begin{frame}
   \frametitle{\small 161109\_1/nocut\_Wen\_fatjet1PrunedML2L3}
   \centering
   \includegraphics[width=0.7\linewidth]{161109_1/nocut_Wen_fatjet1PrunedML2L3.pdf}
\end{frame}

\begin{frame}
   \frametitle{\small 161109\_1/full\_Wmn\_fatjet1PrunedML2L3}
   \centering
   \includegraphics[width=0.7\linewidth]{161109_1/full_Wmn_fatjet1PrunedML2L3.pdf}
\end{frame}

\begin{frame}
   \frametitle{\small 161109\_1/tight\_Wmn\_fatjet1PrunedML2L3}
   \centering
   \includegraphics[width=0.7\linewidth]{161109_1/tight_Wmn_fatjet1PrunedML2L3.pdf}
\end{frame}

\begin{frame}
   \frametitle{\small 161109\_1/nocut\_Wmn\_fatjet1PrunedML2L3}
   \centering
   \includegraphics[width=0.7\linewidth]{161109_1/nocut_Wmn_fatjet1PrunedML2L3.pdf}
\end{frame}

\begin{frame}
   \frametitle{\small 161026/photon\_nocut\_fatjet1PrunedML2L3}
   \centering
   \includegraphics[width=0.7\linewidth]{161026/photon_nocut_fatjet1PrunedML2L3.pdf}
\end{frame}

\begin{frame}
   \frametitle{\small 161027/photon\_nocut\_fatjet1PrunedML2L3}
   \centering
   \includegraphics[width=0.7\linewidth]{161027/photon_nocut_fatjet1PrunedML2L3.pdf}
\end{frame}

\begin{frame}
   \frametitle{\small 161028/photon\_nocut\_fatjet1PrunedML2L3}
   \centering
   \includegraphics[width=0.7\linewidth]{161028/photon_nocut_fatjet1PrunedML2L3.pdf}
\end{frame}

\begin{frame}
   \frametitle{\small 160802/dilep\_nocut\_fatjet1PrunedML2L3}
   \centering
   \includegraphics[width=0.7\linewidth]{160802/dilep_nocut_fatjet1PrunedML2L3.pdf}
\end{frame}

\begin{frame}
   \frametitle{\small 160726/semilep\_full\_0\_0\_fatjetPrunedML2L3}
   \centering
   \includegraphics[width=0.7\linewidth]{160726/semilep_full_0_0_fatjetPrunedML2L3.pdf}
\end{frame}

\begin{frame}
   \frametitle{\small 160726/semilep\_full\_massp\_tau21\_fatjetPrunedML2L3}
   \centering
   \includegraphics[width=0.7\linewidth]{160726/semilep_full_massp_tau21_fatjetPrunedML2L3.pdf}
\end{frame}

\begin{frame}
   \frametitle{\small 160726\_background/semilep\_full\_massp\_tau21\_fatjetPrunedML2L3}
   \centering
   \includegraphics[width=0.7\linewidth]{160726_background/semilep_full_massp_tau21_fatjetPrunedML2L3.pdf}
\end{frame}

\begin{frame}
   \frametitle{\small 160726\_midbackground/semilep\_full\_massp\_tau21\_fatjetPrunedML2L3}
   \centering
   \includegraphics[width=0.7\linewidth]{160726_midbackground/semilep_full_massp_tau21_fatjetPrunedML2L3.pdf}
\end{frame}

\begin{frame}
   \frametitle{\small 160726\_morebackground/semilep\_full\_massp\_tau21\_fatjetPrunedML2L3}
   \centering
   \includegraphics[width=0.7\linewidth]{160726_morebackground/semilep_full_massp_tau21_fatjetPrunedML2L3.pdf}
\end{frame}

\begin{frame}
   \frametitle{\small 160726/semilep\_full\_0\_1\_fatjetPrunedML2L3}
   \centering
   \includegraphics[width=0.7\linewidth]{160726/semilep_full_0_1_fatjetPrunedML2L3.pdf}
\end{frame}

\begin{frame}
   \frametitle{\small 160726/semilep\_full\_0\_3\_fatjetPrunedML2L3}
   \centering
   \includegraphics[width=0.7\linewidth]{160726/semilep_full_0_3_fatjetPrunedML2L3.pdf}
\end{frame}

\begin{frame}
   \frametitle{\small 160726\_background/semilep\_full\_0\_3\_fatjetPrunedML2L3}
   \centering
   \includegraphics[width=0.7\linewidth]{160726_background/semilep_full_0_3_fatjetPrunedML2L3.pdf}
\end{frame}

\begin{frame}
   \frametitle{\small 160726/semilep\_full\_0\_5\_fatjetPrunedML2L3}
   \centering
   \includegraphics[width=0.7\linewidth]{160726/semilep_full_0_5_fatjetPrunedML2L3.pdf}
\end{frame}

\begin{frame}
   \frametitle{\small 160726\_background/semilep\_full\_0\_5\_fatjetPrunedML2L3}
   \centering
   \includegraphics[width=0.7\linewidth]{160726_background/semilep_full_0_5_fatjetPrunedML2L3.pdf}
\end{frame}

\begin{frame}
   \frametitle{\small 160726/semilep\_full\_0\_7\_fatjetPrunedML2L3}
   \centering
   \includegraphics[width=0.7\linewidth]{160726/semilep_full_0_7_fatjetPrunedML2L3.pdf}
\end{frame}

\begin{frame}
   \frametitle{\small 160726\_background/semilep\_full\_fatjetPrunedML2L3}
   \centering
   \includegraphics[width=0.7\linewidth]{160726_background/semilep_full_fatjetPrunedML2L3.pdf}
\end{frame}

\begin{frame}
   \frametitle{\small 160726\_midbackground/semilep\_full\_fatjetPrunedML2L3}
   \centering
   \includegraphics[width=0.7\linewidth]{160726_midbackground/semilep_full_fatjetPrunedML2L3.pdf}
\end{frame}

\begin{frame}
   \frametitle{\small 160726\_morebackground/semilep\_full\_fatjetPrunedML2L3}
   \centering
   \includegraphics[width=0.7\linewidth]{160726_morebackground/semilep_full_fatjetPrunedML2L3.pdf}
\end{frame}

\begin{frame}
   \frametitle{\small 160727\_down/semilep\_full\_fatjetPrunedML2L3}
   \centering
   \includegraphics[width=0.7\linewidth]{160727_down/semilep_full_fatjetPrunedML2L3.pdf}
\end{frame}

\begin{frame}
   \frametitle{\small 160726\_up/semilep\_full\_fatjetPrunedML2L3}
   \centering
   \includegraphics[width=0.7\linewidth]{160726_up/semilep_full_fatjetPrunedML2L3.pdf}
\end{frame}

\begin{frame}
   \frametitle{\small 160726/semilep\_nocut\_nsmalljets\_fatjetPrunedML2L3}
   \centering
   \includegraphics[width=0.7\linewidth]{160726/semilep_nocut_nsmalljets_fatjetPrunedML2L3.pdf}
\end{frame}

\begin{frame}
   \frametitle{\small 160726/semilep\_full\_highpt\_fatjetPrunedML2L3}
   \centering
   \includegraphics[width=0.7\linewidth]{160726/semilep_full_highpt_fatjetPrunedML2L3.pdf}
\end{frame}

\begin{frame}
   \frametitle{\small 160726/semilep\_full\_massp\_tau21\_fatjetDRLooseB}
   \centering
   \includegraphics[width=0.7\linewidth]{160726/semilep_full_massp_tau21_fatjetDRLooseB.pdf}
\end{frame}

\begin{frame}
   \frametitle{\small 160726/semilep\_full\_fatjetDRLooseB}
   \centering
   \includegraphics[width=0.7\linewidth]{160726/semilep_full_fatjetDRLooseB.pdf}
\end{frame}

\begin{frame}
   \frametitle{\small 160726/semilep\_nocut\_nsmalljets\_fatjetDRLooseB}
   \centering
   \includegraphics[width=0.7\linewidth]{160726/semilep_nocut_nsmalljets_fatjetDRLooseB.pdf}
\end{frame}

\begin{frame}
   \frametitle{\small 160726/semilep\_full\_highpt\_fatjetDRLooseB}
   \centering
   \includegraphics[width=0.7\linewidth]{160726/semilep_full_highpt_fatjetDRLooseB.pdf}
\end{frame}

\begin{frame}
   \frametitle{\small 160726/semilep\_nocut\_fatjetDRLooseB}
   \centering
   \includegraphics[width=0.7\linewidth]{160726/semilep_nocut_fatjetDRLooseB.pdf}
\end{frame}

\begin{frame}
   \frametitle{\small met\_SR}
   \centering
   \includegraphics[width=0.7\linewidth]{met_SR.pdf}
\end{frame}

\begin{frame}
   \frametitle{\small 161026/photon\_full\_recoil}
   \centering
   \includegraphics[width=0.7\linewidth]{161026/photon_full_recoil.pdf}
\end{frame}

\begin{frame}
   \frametitle{\small 161027/photon\_full\_recoil}
   \centering
   \includegraphics[width=0.7\linewidth]{161027/photon_full_recoil.pdf}
\end{frame}

\begin{frame}
   \frametitle{\small 161028/photon\_full\_recoil}
   \centering
   \includegraphics[width=0.7\linewidth]{161028/photon_full_recoil.pdf}
\end{frame}

\begin{frame}
   \frametitle{\small 161026/photon\_nocut\_recoil}
   \centering
   \includegraphics[width=0.7\linewidth]{161026/photon_nocut_recoil.pdf}
\end{frame}

\begin{frame}
   \frametitle{\small 161027/photon\_nocut\_recoil}
   \centering
   \includegraphics[width=0.7\linewidth]{161027/photon_nocut_recoil.pdf}
\end{frame}

\begin{frame}
   \frametitle{\small 161028/photon\_nocut\_recoil}
   \centering
   \includegraphics[width=0.7\linewidth]{161028/photon_nocut_recoil.pdf}
\end{frame}

\begin{frame}
   \frametitle{\small 160726/semilep\_full\_massp\_tau21\_n\_bjetsMedium}
   \centering
   \includegraphics[width=0.7\linewidth]{160726/semilep_full_massp_tau21_n_bjetsMedium.pdf}
\end{frame}

\begin{frame}
   \frametitle{\small 160726/semilep\_full\_n\_bjetsMedium}
   \centering
   \includegraphics[width=0.7\linewidth]{160726/semilep_full_n_bjetsMedium.pdf}
\end{frame}

\begin{frame}
   \frametitle{\small 160726/semilep\_full\_highpt\_n\_bjetsMedium}
   \centering
   \includegraphics[width=0.7\linewidth]{160726/semilep_full_highpt_n_bjetsMedium.pdf}
\end{frame}

\begin{frame}
   \frametitle{\small 160726/semilep\_nocut\_n\_bjetsMedium}
   \centering
   \includegraphics[width=0.7\linewidth]{160726/semilep_nocut_n_bjetsMedium.pdf}
\end{frame}

\begin{frame}
   \frametitle{\small ttselection}
   \centering
   \includegraphics[width=0.7\linewidth]{ttselection.pdf}
\end{frame}

\begin{frame}
   \frametitle{\small 160726/semilep\_full\_massp\_tau21\_fatjetMass}
   \centering
   \includegraphics[width=0.7\linewidth]{160726/semilep_full_massp_tau21_fatjetMass.pdf}
\end{frame}

\begin{frame}
   \frametitle{\small 160726/semilep\_full\_fatjetMass}
   \centering
   \includegraphics[width=0.7\linewidth]{160726/semilep_full_fatjetMass.pdf}
\end{frame}

\begin{frame}
   \frametitle{\small 160726/semilep\_full\_highpt\_fatjetMass}
   \centering
   \includegraphics[width=0.7\linewidth]{160726/semilep_full_highpt_fatjetMass.pdf}
\end{frame}

\begin{frame}
   \frametitle{\small 160726\_background/smearedup\_mass}
   \centering
   \includegraphics[width=0.7\linewidth]{160726_background/smearedup_mass.pdf}
\end{frame}

\begin{frame}
   \frametitle{\small 161026/photon\_full\_fatjet1Pt}
   \centering
   \includegraphics[width=0.7\linewidth]{161026/photon_full_fatjet1Pt.pdf}
\end{frame}

\begin{frame}
   \frametitle{\small 161027/photon\_full\_fatjet1Pt}
   \centering
   \includegraphics[width=0.7\linewidth]{161027/photon_full_fatjet1Pt.pdf}
\end{frame}

\begin{frame}
   \frametitle{\small 161026/photon\_nocut\_fatjet1Pt}
   \centering
   \includegraphics[width=0.7\linewidth]{161026/photon_nocut_fatjet1Pt.pdf}
\end{frame}

\begin{frame}
   \frametitle{\small 161027/photon\_nocut\_fatjet1Pt}
   \centering
   \includegraphics[width=0.7\linewidth]{161027/photon_nocut_fatjet1Pt.pdf}
\end{frame}

\begin{frame}
   \frametitle{\small 160726/semilep\_full\_massp\_tau21\_fatjetPt}
   \centering
   \includegraphics[width=0.7\linewidth]{160726/semilep_full_massp_tau21_fatjetPt.pdf}
\end{frame}

\begin{frame}
   \frametitle{\small 160726/semilep\_full\_fatjetPt}
   \centering
   \includegraphics[width=0.7\linewidth]{160726/semilep_full_fatjetPt.pdf}
\end{frame}

\begin{frame}
   \frametitle{\small 160726/semilep\_full\_massp\_tau21\_n\_jetsNotFat}
   \centering
   \includegraphics[width=0.7\linewidth]{160726/semilep_full_massp_tau21_n_jetsNotFat.pdf}
\end{frame}

\begin{frame}
   \frametitle{\small 160726/semilep\_full\_n\_jetsNotFat}
   \centering
   \includegraphics[width=0.7\linewidth]{160726/semilep_full_n_jetsNotFat.pdf}
\end{frame}

\begin{frame}
   \frametitle{\small 160726/semilep\_full\_highpt\_n\_jetsNotFat}
   \centering
   \includegraphics[width=0.7\linewidth]{160726/semilep_full_highpt_n_jetsNotFat.pdf}
\end{frame}

\begin{frame}
   \frametitle{\small 160726/semilep\_nocut\_n\_jetsNotFat}
   \centering
   \includegraphics[width=0.7\linewidth]{160726/semilep_nocut_n_jetsNotFat.pdf}
\end{frame}

\begin{frame}
   \frametitle{\small 160726/semilep\_full\_massp\_tau21\_met}
   \centering
   \includegraphics[width=0.7\linewidth]{160726/semilep_full_massp_tau21_met.pdf}
\end{frame}

\begin{frame}
   \frametitle{\small 161109\_1/full\_Zee\_met}
   \centering
   \includegraphics[width=0.7\linewidth]{161109_1/full_Zee_met.pdf}
\end{frame}

\begin{frame}
   \frametitle{\small 161109\_1/tight\_Zee\_met}
   \centering
   \includegraphics[width=0.7\linewidth]{161109_1/tight_Zee_met.pdf}
\end{frame}

\begin{frame}
   \frametitle{\small 161109\_1/nocut\_Zee\_met}
   \centering
   \includegraphics[width=0.7\linewidth]{161109_1/nocut_Zee_met.pdf}
\end{frame}

\begin{frame}
   \frametitle{\small 161109\_1/full\_signal\_met}
   \centering
   \includegraphics[width=0.7\linewidth]{161109_1/full_signal_met.pdf}
\end{frame}

\begin{frame}
   \frametitle{\small 161109\_1/tight\_signal\_met}
   \centering
   \includegraphics[width=0.7\linewidth]{161109_1/tight_signal_met.pdf}
\end{frame}

\begin{frame}
   \frametitle{\small 161109\_1/nocut\_signal\_met}
   \centering
   \includegraphics[width=0.7\linewidth]{161109_1/nocut_signal_met.pdf}
\end{frame}

\begin{frame}
   \frametitle{\small 160726/semilep\_full\_met}
   \centering
   \includegraphics[width=0.7\linewidth]{160726/semilep_full_met.pdf}
\end{frame}

\begin{frame}
   \frametitle{\small 161109\_1/full\_Zmm\_met}
   \centering
   \includegraphics[width=0.7\linewidth]{161109_1/full_Zmm_met.pdf}
\end{frame}

\begin{frame}
   \frametitle{\small 161109\_1/tight\_Zmm\_met}
   \centering
   \includegraphics[width=0.7\linewidth]{161109_1/tight_Zmm_met.pdf}
\end{frame}

\begin{frame}
   \frametitle{\small 161109\_1/nocut\_Zmm\_met}
   \centering
   \includegraphics[width=0.7\linewidth]{161109_1/nocut_Zmm_met.pdf}
\end{frame}

\begin{frame}
   \frametitle{\small 161109\_1/full\_Wen\_met}
   \centering
   \includegraphics[width=0.7\linewidth]{161109_1/full_Wen_met.pdf}
\end{frame}

\begin{frame}
   \frametitle{\small 161109\_1/tight\_Wen\_met}
   \centering
   \includegraphics[width=0.7\linewidth]{161109_1/tight_Wen_met.pdf}
\end{frame}

\begin{frame}
   \frametitle{\small 161109\_1/nocut\_Wen\_met}
   \centering
   \includegraphics[width=0.7\linewidth]{161109_1/nocut_Wen_met.pdf}
\end{frame}

\begin{frame}
   \frametitle{\small 161109\_1/full\_Wmn\_met}
   \centering
   \includegraphics[width=0.7\linewidth]{161109_1/full_Wmn_met.pdf}
\end{frame}

\begin{frame}
   \frametitle{\small 161109\_1/tight\_Wmn\_met}
   \centering
   \includegraphics[width=0.7\linewidth]{161109_1/tight_Wmn_met.pdf}
\end{frame}

\begin{frame}
   \frametitle{\small 161109\_1/nocut\_Wmn\_met}
   \centering
   \includegraphics[width=0.7\linewidth]{161109_1/nocut_Wmn_met.pdf}
\end{frame}

\begin{frame}
   \frametitle{\small 161109\_1/full\_gjets\_met}
   \centering
   \includegraphics[width=0.7\linewidth]{161109_1/full_gjets_met.pdf}
\end{frame}

\begin{frame}
   \frametitle{\small 161109\_1/tight\_gjets\_met}
   \centering
   \includegraphics[width=0.7\linewidth]{161109_1/tight_gjets_met.pdf}
\end{frame}

\begin{frame}
   \frametitle{\small 161109\_1/nocut\_gjets\_met}
   \centering
   \includegraphics[width=0.7\linewidth]{161109_1/nocut_gjets_met.pdf}
\end{frame}

\begin{frame}
   \frametitle{\small 160726/semilep\_nocut\_nsmalljets\_met}
   \centering
   \includegraphics[width=0.7\linewidth]{160726/semilep_nocut_nsmalljets_met.pdf}
\end{frame}

\begin{frame}
   \frametitle{\small 160726/semilep\_full\_highpt\_met}
   \centering
   \includegraphics[width=0.7\linewidth]{160726/semilep_full_highpt_met.pdf}
\end{frame}

\begin{frame}
   \frametitle{\small 160726/semilep\_full\_massp\_tau21\_n\_bjetsTight}
   \centering
   \includegraphics[width=0.7\linewidth]{160726/semilep_full_massp_tau21_n_bjetsTight.pdf}
\end{frame}

\begin{frame}
   \frametitle{\small 160726/semilep\_full\_n\_bjetsTight}
   \centering
   \includegraphics[width=0.7\linewidth]{160726/semilep_full_n_bjetsTight.pdf}
\end{frame}

\begin{frame}
   \frametitle{\small 160726/semilep\_full\_highpt\_n\_bjetsTight}
   \centering
   \includegraphics[width=0.7\linewidth]{160726/semilep_full_highpt_n_bjetsTight.pdf}
\end{frame}

\begin{frame}
   \frametitle{\small 160707/WPt\_comparisonx}
   \centering
   \includegraphics[width=0.7\linewidth]{160707/WPt_comparisonx.pdf}
\end{frame}



\backupend

\end{document}
